%----------------------------------------------------------------------------------------
%	PART - Animation
%----------------------------------------------------------------------------------------

\makeatletter\@openrightfalse
\part{Animation}

\chaptertypein{
	\keybackgroundcolor{gray}
	\keytextcolor{black}
	10 PRINT "\shiftkey\clrhomekey"\\
	20 FOR X=1 TO 38\\
	30 PRINT "\clrhomekey";\\
	40 PRINT TAB(X);"*"\\
	50 FOR D=1 TO 50:NEXT D\\
	60 NEXT X\\
	70 GOTO 20
}

%----------------------------------------------------------------------------------------
%	CHAPTER - Animation Introduction
%----------------------------------------------------------------------------------------

\chapter*{Animation}\index{Animation}
\addcontentsline{toc}{chapter}{\protect\numberline{}Animation}

Did you try the program at the beginning of this chapter?  If you did, you saw
a little asterisk zooming across the screen, over and over again!  Animation is
one of the most exciting things you can do with your Commander X16.  In this
chapter, you'll learn how to make objects move across the screen, bounce
around, and even fly!\\

Animation on a computer works just like animation in cartoons.  A cartoon movie
is really just a series of still pictures shown very quickly, one after
another.  Each picture is slightly different from the one before it, and when
they flash by fast enough, your eyes see smooth movement.  Computer animation
works the same way --- we draw something, erase it, and draw it again in a
slightly different position.\\

If you've read Chapter 3, you already know how to put colors and characters on
the screen.  Now we're going to make those characters \emph{move}!\\

%----------------------------------------------------------------------------------------
%	CHAPTER - Your First Animation
%----------------------------------------------------------------------------------------

\chapter*{Your First Animation}\index{Animation!First Animation}
\addcontentsline{toc}{chapter}{\protect\numberline{}Your First Animation}

Let's take a close look at the program from the beginning of this chapter to
understand how it makes the asterisk move across the screen:\\

\codeblock{
	10 PRINT "\shiftkey\clrhomekey"\\
	20 FOR X=1 TO 38\\
	30 PRINT "\clrhomekey";\\
	40 PRINT TAB(X);"*"\\
	50 FOR D=1 TO 50:NEXT D\\
	60 NEXT X\\
	70 GOTO 20\\
}

Line 10 clears the screen at the start of the program.  This gives us a nice
blank canvas to work with.\\

Line 20 starts a {\ttfamily FOR...NEXT} loop.  The variable {\ttfamily X} will
count from 1 to 38, and we'll use this number to control where the asterisk
appears.  Why 38?  Because the standard screen is 40 characters wide, and we
want to leave a little room at the edges.\\

Line 30 moves the cursor to the home position (top-left corner) without
clearing the screen.  The \clrhomekey key by itself (without \shiftkey) just
moves the cursor home.  The semicolon at the end keeps the cursor on that
line.\\

Line 40 is where the magic happens!  The {\ttfamily TAB(X)} function moves to
column {\ttfamily X}, and then we print the asterisk.  Each time through the
loop, {\ttfamily X} is one bigger, so the asterisk appears one position further
to the right.\\

Line 50 is a \emph{delay loop}.  It counts from 1 to 50, doing nothing useful
--- just wasting a little time.  Without this delay, the asterisk would zoom
across the screen so fast you could barely see it!\\

Line 60 completes the {\ttfamily FOR...NEXT} loop, sending us back to line 20
with a new value of {\ttfamily X}.\\

Line 70 sends us back to line 20 to start the whole animation over again.\\

\note{
	The delay loop in line 50 is a classic technique from early computers.  The
	Commander X16 also has a {\ttfamily SLEEP} command that pauses for a
	specified number of ``jiffies'' (60ths of a second), which can be easier to
	use.  Try replacing line 50 with {\ttfamily 50 SLEEP 1} and see if you like
	the timing better!\\
}

\section{Understanding the Animation}

Here's what happens each time through the loop:\\

\begin{enumerate}
	\item The cursor moves to the home position (top-left corner)
	\item The {\ttfamily TAB} function moves to column X
	\item The asterisk is printed at that position
	\item The delay loop pauses briefly
	\item X increases by 1, and we repeat
\end{enumerate}

But wait --- why don't we see a trail of asterisks across the screen?  Look
carefully at line 30 again.  The \clrhomekey moves the cursor to the top-left,
and then when we print the asterisk with {\ttfamily TAB(X)}, the spaces from
column 1 to column X automatically erase any previous asterisk!  The {\ttfamily
TAB} function fills in spaces as it moves across.\\

\tryit{
	Change line 40 to use a different character instead of the asterisk.  Try a
	ball ({\ttfamily \shiftkey\key{Q}}), a heart ({\ttfamily \shiftkey\key{S}}),
	or your favorite graphic character from Chapter 3!\\
}

%----------------------------------------------------------------------------------------
%	CHAPTER - The Bouncing Ball
%----------------------------------------------------------------------------------------

\chapter*{The Bouncing Ball}\index{Animation!Bouncing Ball}
\addcontentsline{toc}{chapter}{\protect\numberline{}The Bouncing Ball}

Now let's make something that moves in \emph{two} directions --- the classic
bouncing ball!  Type {\ttfamily NEW} and enter this program:\\

\codeblock{
	10 PRINT "\shiftkey\clrhomekey"\\
	20 X=1:Y=1\\
	30 DX=1:DY=1\\
	40 TILE X,Y,81\\
	50 SLEEP 2\\
	60 TILE X,Y,32\\
	70 X=X+DX:Y=Y+DY\\
	80 IF X<1 OR X>38 THEN DX=-DX\\
	90 IF Y<1 OR Y>28 THEN DY=-DY\\
	100 GOTO 40\\
}

Run this program and watch the ball bounce around the screen!  It moves
diagonally, and when it hits an edge, it bounces back.  Press \runstopkey to
stop the program when you've seen enough.\\

Let's look at how this program works:\\

Lines 20-30 set up our \emph{variables}.  {\ttfamily X} and {\ttfamily Y} are
the ball's position on the screen.  {\ttfamily DX} and {\ttfamily DY} are the
ball's \emph{direction} --- how much to move each step.  When {\ttfamily DX} is
1, the ball moves right; when it's -1, the ball moves left.  The same is true
for {\ttfamily DY} and up/down movement.\\

Line 40 draws the ball using the {\ttfamily TILE} statement.  The number 81 is
the screen code for a filled circle ({\ttfamily \shiftkey\key{Q}}).  Remember
from Chapter 3 that {\ttfamily TILE} takes X position, Y position, and
character code.\\

Line 50 pauses for a tiny moment so we can see the ball.  {\ttfamily SLEEP 2}
waits for 2 ``jiffies'' (about 1/30th of a second).\\

Line 60 erases the ball by putting a space (screen code 32) at the ball's
current position.  This is important!  If we didn't erase the old ball, we'd
see a trail of balls instead of one moving ball.\\

Line 70 moves the ball by adding the direction values to the position.  If
{\ttfamily DX} is 1, then {\ttfamily X=X+DX} makes {\ttfamily X} one bigger,
moving the ball to the right.\\

Lines 80-90 check if the ball has hit an edge.  If {\ttfamily X} is less than 1
or greater than 38, we've hit the left or right edge, so we reverse the
horizontal direction by making {\ttfamily DX=-DX}.  The same logic applies for
the top and bottom edges with {\ttfamily Y} and {\ttfamily DY}.\\

Line 100 goes back to line 40 to draw the ball in its new position and continue
the animation.\\

\section{Adding Color to the Ball}

Let's make the bouncing ball more colorful!  Modify line 40 to include a
color:\\

\codeblock{
	40 TILE X,Y,81,\$71\\
}

The {\ttfamily \$71} sets the color to white (1) on a yellow (7) background.
But why stop at one color?  Let's make the ball change colors as it bounces!\\

\codeblock{
	10 PRINT "\shiftkey\clrhomekey"\\
	20 X=1:Y=1:C=1\\
	30 DX=1:DY=1\\
	40 TILE X,Y,81,C*16+C\\
	50 SLEEP 2\\
	60 TILE X,Y,32\\
	70 X=X+DX:Y=Y+DY\\
	80 IF X<1 OR X>38 THEN DX=-DX:C=C+1\\
	90 IF Y<1 OR Y>28 THEN DY=-DY:C=C+1\\
	95 IF C>15 THEN C=1\\
	100 GOTO 40\\
}

Now the ball changes color every time it bounces!  Line 95 makes sure the color
stays in the valid range of 1-15 (we skip 0, which is black on black).\\

\tip{Making it Faster or Slower}{
	The {\ttfamily SLEEP 2} in line 50 controls the speed of the animation.
	Change it to {\ttfamily SLEEP 1} for a faster ball, or {\ttfamily SLEEP 5}
	for a slower one.  You can even remove line 50 entirely for maximum speed
	--- but the ball might move too fast to follow!\\
}

%----------------------------------------------------------------------------------------
%	CHAPTER - Cursor Control for Animation
%----------------------------------------------------------------------------------------

\chapter*{Cursor Control for Animation}\index{Animation!Cursor Control}\index{Cursor Control}
\addcontentsline{toc}{chapter}{\protect\numberline{}Cursor Control for Animation}

Before the {\ttfamily TILE} statement existed, early Commodore programmers
animated by using cursor control characters inside {\ttfamily PRINT} statements.
This technique still works on the Commander X16, and it's a fun way to
understand how animation really works.\\

\section{Cursor Movement Characters}

You can move the cursor around the screen using special control characters.
These can be typed inside quotation marks in a {\ttfamily PRINT} statement:\\

\begin{tabular}{|c|l|}
	\hline
	{\bfseries Keys} & {\bfseries Action}\\ \hline
	\ctrlkey + \key{N} or Cursor Down & Move cursor down one row\\ \hline
	\ctrlkey + \key{P} or Cursor Up & Move cursor up one row\\ \hline
	\ctrlkey + \key{M} or Cursor Right & Move cursor right one column\\ \hline
	\ctrlkey + \key{Y} or Cursor Left & Move cursor left one column\\ \hline
	\clrhomekey (alone) & Move cursor to home position\\ \hline
	\shiftkey + \clrhomekey & Clear screen and home cursor\\ \hline
\end{tabular}

\vspace{16pt}

When you press these keys while inside quotation marks in a {\ttfamily PRINT}
statement, a special symbol appears in your program listing.  When the program
runs, the Commander X16 interprets that symbol as a cursor movement
instruction.\\

\section{A Moving Star with Cursor Control}

Here's a program that moves a star across the screen using cursor control:\\

\codeblock{
	10 PRINT "\shiftkey\clrhomekey"\\
	20 PRINT "\key{CrsrDn}\key{CrsrDn}\key{CrsrDn}\key{CrsrDn}\key{CrsrDn}";\\
	30 FOR I=1 TO 35\\
	40 PRINT " *\key{CrsrLt}";\\
	50 FOR D=1 TO 50:NEXT D\\
	60 NEXT I\\
	70 GOTO 10\\
}

In line 20, you would press the cursor-down key five times while inside the
quotation marks.  This moves the cursor down five rows so the star doesn't
appear at the very top of the screen.\\

Line 40 is where the animation magic happens.  Inside the quotation marks,
we print:\\
\begin{itemize}
	\item A space (to erase the previous star)
	\item An asterisk (the new star)
	\item A cursor-left (to position for the next erase)
\end{itemize}

Each time through the loop, the space erases the old star, a new star is
printed one position to the right, and the cursor-left gets us ready for the
next cycle.\\

\note{
	When typing cursor control characters inside quotation marks, you'll see
	strange-looking symbols appear.  Don't worry --- this is normal!  These
	symbols tell the Commander X16 to move the cursor when the program runs.
	They'll look like reversed letters or special graphics characters.\\
}

\section{The Classic Bouncing Ball with Cursor Control}

Here's the traditional way programmers made a bouncing ball move across the
screen, using only cursor control and {\ttfamily PRINT} statements:\\

\codeblock{
	10 REM BOUNCING BALL\\
	20 PRINT "\shiftkey\clrhomekey"\\
	25 FOR X=1 TO 10:PRINT "\key{CrsrDn}":NEXT\\
	30 FOR BL=1 TO 38\\
	40 PRINT " \shiftkey\key{Q}\key{CrsrLt}";\\
	50 FOR TM=1 TO 50:NEXT TM\\
	60 NEXT BL\\
	70 REM MOVE BALL RIGHT TO LEFT\\
	80 FOR BL=38 TO 1 STEP -1\\
	90 PRINT " \key{CrsrLt}\key{CrsrLt}\shiftkey\key{Q}\key{CrsrLt}";\\
	100 FOR TM=1 TO 50:NEXT TM\\
	110 NEXT BL\\
	120 GOTO 20\\
}

This program moves a ball from left to right, then from right to left, over and
over.  Line 25 positions the ball in the middle of the screen vertically.
Lines 30-60 move the ball to the right, and lines 80-110 move it back to the
left.\\

The trick in line 40 is printing a space first (to erase the old ball), then
the ball character (\shiftkey + \key{Q}), and finally a cursor-left to position
for the next erase.\\

Line 90 is a bit different because we're moving left.  We print a space, then
two cursor-lefts (to move past where the ball was), the ball, and then one more
cursor-left.\\

\tryit{
	Try changing line 25 to position the ball at different heights on the
	screen.  What happens if you use more or fewer cursor-down characters?  Can
	you make the ball bounce along the top of the screen?\\
}

%----------------------------------------------------------------------------------------
%	CHAPTER - Animation with VPOKE
%----------------------------------------------------------------------------------------

\chapter*{Animation with VPOKE}\index{Animation!VPOKE}\index{VPOKE}
\addcontentsline{toc}{chapter}{\protect\numberline{}Animation with VPOKE}

For the fastest possible animation, you can write directly to the Commander
X16's video memory using the {\ttfamily VPOKE} command.  This bypasses the
normal {\ttfamily PRINT} routines and lets you put characters and colors
anywhere on the screen almost instantly.\\

The Commander X16's video chip is called VERA (Video Enhanced Retro Adapter).
VERA has its own memory that holds the screen contents, and {\ttfamily VPOKE}
lets you write directly to that memory.\\

\section{Understanding Screen Memory}

In the default text mode, each character on screen takes up two bytes in VERA
memory:\\

\begin{itemize}
	\item The first byte is the \emph{screen code} --- which character to display
	\item The second byte is the \emph{color} --- foreground and background colors
\end{itemize}

The screen memory starts at address {\ttfamily \$1B000} in VERA memory, which
means bank 1 at address {\ttfamily \$B000}.  Each row of 80 characters takes up
160 bytes (80 characters × 2 bytes each).  To find the address of any character
at column X and row Y, use this formula:\\

\begin{center}
	{\ttfamily ADDRESS = \$B000 + (Y * 160) + (X * 2)}
\end{center}

The color byte is at ADDRESS + 1.\\

\section{Using VPOKE}

The {\ttfamily VPOKE} command takes three numbers: the VERA bank, the address,
and the value to write.  For the text screen, we use bank 1 with addresses
starting at {\ttfamily \$B000}:\\

\codeblock{
	VPOKE 1, \$B000 + offset, value\\
}

Here's a simple example that puts a red ``A'' at the top-left corner of the
screen:\\

\codeblock{
	10 VPOKE 1,\$B000,1\\
	20 VPOKE 1,\$B001,\$02\\
}

Line 10 puts screen code 1 (the letter A) at address {\ttfamily \$B000} in bank
1.  Line 20 puts color {\ttfamily \$02} (red foreground, black background) at
address {\ttfamily \$B001}.\\

\section{A Fast Bouncing Ball}

Here's a bouncing ball program using {\ttfamily VPOKE} for faster animation:\\

\codeblock{
	10 PRINT "\shiftkey\clrhomekey"\\
	20 X=0:Y=0\\
	30 DX=1:DY=1\\
	40 A=\$B000+(Y*160)+(X*2)\\
	50 VPOKE 1,A,81:VPOKE 1,A+1,\$71\\
	60 SLEEP 1\\
	70 VPOKE 1,A,32\\
	80 X=X+DX:Y=Y+DY\\
	90 IF X<0 OR X>78 THEN DX=-DX\\
	100 IF Y<0 OR Y>58 THEN DY=-DY\\
	110 GOTO 40\\
}

This program works just like our earlier bouncing ball, but uses {\ttfamily
VPOKE} instead of {\ttfamily TILE}.  Line 40 calculates the memory address for
position X,Y.  Line 50 writes both the character (81 = ball) and the color
(\$71 = white on yellow).  Line 70 erases the ball by writing a space (32).\\

\note{
	The {\ttfamily TILE} statement is usually easier to use than {\ttfamily
	VPOKE}, but {\ttfamily VPOKE} gives you more control and can be slightly
	faster for complex animations.  When you're starting out, stick with
	{\ttfamily TILE}.  As you become more advanced, you might want to experiment
	with {\ttfamily VPOKE} for special effects.\\
}

%----------------------------------------------------------------------------------------
%	CHAPTER - Flying Objects
%----------------------------------------------------------------------------------------

\chapter*{Flying Objects}\index{Animation!Flying Objects}
\addcontentsline{toc}{chapter}{\protect\numberline{}Flying Objects}

Let's put everything together and create a program with multiple moving
objects!  We'll make a scene with a flying bird and some clouds.\\

\section{A Simple Flying Bird}

First, let's make a bird that flaps its wings as it flies.  The trick is to
alternate between two different characters to create the flapping effect:\\

\codeblock{
	10 PRINT "\shiftkey\clrhomekey"\\
	20 FOR X=0 TO 78\\
	30 W=(X AND 1)\\
	40 IF W=0 THEN TILE X,10,118\\
	50 IF W=1 THEN TILE X,10,119\\
	60 SLEEP 3\\
	70 TILE X,10,32\\
	80 NEXT X\\
	90 GOTO 10\\
}

Screen codes 118 and 119 look like a bird with wings up and wings down.  Line
30 uses {\ttfamily AND 1} to check if X is odd or even --- this makes the wings
alternate with each step.  Lines 40-50 draw the appropriate wing position, and
line 70 erases the bird before it moves to the next position.\\

\section{Adding Multiple Birds}

Now let's make things more interesting with several birds flying at different
heights and speeds:\\

\codeblock{
	10 PRINT "\shiftkey\clrhomekey"\\
	20 DIM X(3),Y(3),S(3)\\
	30 FOR I=0 TO 2\\
	40 X(I)=RND(1)*40\\
	50 Y(I)=RND(1)*10+5\\
	60 S(I)=INT(RND(1)*3)+1\\
	70 NEXT I\\
	80 REM ANIMATION LOOP\\
	90 FOR I=0 TO 2\\
	100 TILE X(I),Y(I),32\\
	110 X(I)=X(I)+S(I)\\
	120 IF X(I)>78 THEN X(I)=0\\
	130 W=X(I) AND 1\\
	140 IF W=0 THEN TILE X(I),Y(I),118\\
	150 IF W=1 THEN TILE X(I),Y(I),119\\
	160 NEXT I\\
	170 SLEEP 2\\
	180 GOTO 90\\
}

This program uses \emph{arrays} to keep track of three birds.  Each bird has
its own X position ({\ttfamily X()}), Y position ({\ttfamily Y()}), and speed
({\ttfamily S()}).  Lines 30-70 initialize each bird with random starting
positions and speeds.  The main loop at lines 90-160 moves and draws all three
birds.\\

\tryit{
	Change the number of birds by modifying line 20 to {\ttfamily DIM X(5),
	Y(5), S(5)} and changing the loops to go from 0 to 4.  How many birds can
	you have flying at once before the animation starts to slow down?\\
}

\section{A Starfield}

Here's a fun program that creates a scrolling starfield, like you're flying
through space:\\

\codeblock{
	10 PRINT "\shiftkey\clrhomekey"\\
	20 DIM X(20),Y(20),S(20)\\
	30 FOR I=0 TO 19\\
	40 X(I)=RND(1)*80\\
	50 Y(I)=RND(1)*60\\
	60 S(I)=INT(RND(1)*3)+1\\
	70 NEXT I\\
	80 COLOR 1,0\\
	90 REM ANIMATION LOOP\\
	100 FOR I=0 TO 19\\
	110 TILE X(I),Y(I),32\\
	120 X(I)=X(I)-S(I)\\
	130 IF X(I)<0 THEN X(I)=79:Y(I)=RND(1)*60\\
	140 TILE X(I),Y(I),46,\$01\\
	150 NEXT I\\
	160 SLEEP 1\\
	170 GOTO 100\\
}

Twenty ``stars'' (actually periods) move from right to left at different
speeds.  When a star goes off the left edge, it reappears on the right at a
random height.  Stars moving faster appear to be closer, creating a sense of
depth!\\

\section{Putting It All Together}

Here's a more complete program that combines several animation techniques into
a little scene:\\

\codeblock{
	10 PRINT "\shiftkey\clrhomekey"\\
	20 COLOR 14,6\\
	30 LOCATE 25\\
	40 FOR I=1 TO 80:PRINT "\shiftkey\key{-}";:NEXT\\
	50 COLOR 5\\
	60 X=0:Y=10\\
	70 TILE X,Y,118+((X AND 2)/2)\\
	80 SLEEP 2\\
	90 TILE X,Y,32\\
	100 X=X+1\\
	110 IF X>79 THEN X=0:Y=INT(RND(1)*15)+5\\
	120 GOTO 70\\
}

This creates a simple scene with a ``ground'' line (made of horizontal
lines) and a bird flying across the sky.  When the bird goes off the right
edge, it reappears on the left at a random height.\\

\tip{Making Your Own Animations}{
	The key to good animation is practice!  Start with simple programs like
	the ones in this chapter, then try combining them.  Add colors, sounds (from
	Chapter 5), and user input (from Chapter 6) to create your own animated
	programs and games.  The Commander X16 has the power to create amazing
	animations --- all it takes is your imagination!\\
}

\@openrighttrue\makeatother
