%----------------------------------------------------------------------------------------
%	PART - Color and Graphics
%----------------------------------------------------------------------------------------

\makeatletter\@openrightfalse
\part{Color and Graphics}

\chaptertypein{
	\keybackgroundcolor{gray}
	\keytextcolor{black}
	10 COLOR 6,1\\
	20 PRINT "\shiftkey\clrhomekey"\\
	30 FOR I=1 TO 16\\
	40 COLOR I,1\\
	50 PRINT "HELLO X16!"\\
	60 NEXT I
}

%----------------------------------------------------------------------------------------
%	CHAPTER - Color and Graphics
%----------------------------------------------------------------------------------------

\chapter*{Color and Graphics}\index{Color}\index{Graphics}
\addcontentsline{toc}{chapter}{\protect\numberline{}Color and Graphics}

Did you try the program at the beginning of this chapter?  If you did, you saw
the words "HELLO X16!" printed on the screen in many different colors!  The
Commander X16 can display text and graphics in 16 beautiful colors.  In this
chapter, you'll learn how to add color to your programs and create colorful
graphics using the keyboard's special graphic characters.\\

If you've read Chapter 2, you already know a little bit about the {\ttfamily
COLOR} statement and the color keys on the keyboard.  Now we're going to
explore color in much more detail and learn some exciting ways to use it in
your programs.\\

%----------------------------------------------------------------------------------------
%	CHAPTER - Programming in Color
%----------------------------------------------------------------------------------------

\chapter*{Programming in Color}\index{COLOR}
\addcontentsline{toc}{chapter}{\protect\numberline{}Programming in Color}

The {\ttfamily COLOR} statement is the easiest way to add color to your
programs.  When you use the {\ttfamily COLOR} statement, any text you print
after it will appear in the color you chose.  The {\ttfamily COLOR} statement
takes one or two numbers.  The first number sets the \emph{foreground color}
(the color of the letters themselves), and the optional second number sets the
\emph{background color} (the color behind each letter).\\

Let's try a simple example.  Type {\ttfamily NEW} and press \returnkey to clear
any old program, then type:\\

\codeblock{
	10 COLOR 2\\
	20 PRINT "I AM RED!"\\
	30 COLOR 5\\
	40 PRINT "I AM GREEN!"\\
	50 COLOR 7\\
	60 PRINT "I AM YELLOW!"\\
}

When you type {\ttfamily RUN} and press \returnkey, you should see each message
printed in a different color!  The Commander X16 remembers the color you set
and uses it for everything you print until you change it again.\\

\section{The 16 Colors}

The Commander X16 has 16 colors you can use with the {\ttfamily COLOR}
statement.  Here they are:\\

\begin{tabular}{|c|l||c|l|}
	\hline
	{\bfseries Number} & {\bfseries Color} & {\bfseries Number} & {\bfseries Color}\\ \hline
	0 & Black & 8 & Orange \\ \hline
	1 & White & 9 & Brown \\ \hline
	2 & Red & 10 & Light Red \\ \hline
	3 & Cyan & 11 & Dark Gray\\ \hline
	4 & Purple & 12 & Medium Gray\\ \hline
	5 & Green & 13 & Light Green\\ \hline
	6 & Blue & 14 & Light Blue\\ \hline
	7 & Yellow & 15 & Light Gray\\ \hline
\end{tabular}

\vspace{16pt}

\tryit{
	Modify the program above to use different color numbers.  Try using all 16
	colors to see what they look like!\\
}

\section{Foreground and Background Colors}

Each character on the screen can have its own foreground color \emph{and} its
own background color.  This lets you create very colorful displays!  To set
both colors at once, use the {\ttfamily COLOR} statement with two numbers
separated by a comma:\\

\codeblock{
	10 COLOR 1,2\\
	20 PRINT "WHITE ON RED"\\
	30 COLOR 7,6\\
	40 PRINT "YELLOW ON BLUE"\\
	50 COLOR 0,1\\
	60 PRINT "BLACK ON WHITE"\\
}

Try running this program.  Each line of text has a different combination of
foreground and background colors.  Notice how each character has its own little
rectangle of background color --- this is one of the things that makes the
Commander X16 so much fun for creating colorful screens!\\

%----------------------------------------------------------------------------------------
%	CHAPTER - The Color Keys
%----------------------------------------------------------------------------------------

\chapter*{The Color Keys}\index{Color Keys}
\addcontentsline{toc}{chapter}{\protect\numberline{}The Color Keys}

Look at the number keys on your Commander X16 keyboard.  You'll see that each
number from 1 through 8 has two colors printed next to it.  These are the
\emph{color keys}.  By holding down \ctrlkey or \altkey while pressing one of
these number keys, you can change the color of the text you type --- even in
the middle of typing!\\

\begin{tabular}{|c|c|c|}
	\hline
	{\bfseries Key} & {\bfseries CTRL} & {\bfseries ALT}\\ \hline
	1 & Black & Orange \\ \hline
	2 & White & Brown \\ \hline
	3 & Red & Light Red \\ \hline
	4 & Cyan & Dark Gray\\ \hline
	5 & Purple & Medium Gray\\ \hline
	6 & Green & Light Green\\ \hline
	7 & Blue & Light Blue\\ \hline
	8 & Yellow & Light Gray\\ \hline
\end{tabular}

\vspace{16pt}

Holding \ctrlkey while pressing a number key selects the color listed on top.
Holding \altkey while pressing a number key selects the color listed on the
bottom.  This gives you all 16 colors right from the keyboard!\\

\section{Using Color Keys in Programs}

The color keys are especially useful inside {\ttfamily PRINT} statements.
When you're typing the text inside quotation marks, you can press \ctrlkey or
\altkey with a number key to insert a special control character.  This
character will look like a strange symbol on your screen, but when the program
runs, it will change the color of the text that follows it.\\

Try this program:\\

\codeblock{
	10 PRINT "THIS IS \ctrlkey\key{3}RED\ctrlkey\key{6} AND GREEN"\\
}

When you type this line, after typing {\ttfamily THIS IS }, hold down \ctrlkey
and press \key{3}.  You'll see a special character appear.  Then type
{\ttfamily RED}, hold \ctrlkey and press \key{6}, and finally type
{\ttfamily\ AND GREEN"}.  When you run the program, the word "RED" will appear
in red and "AND GREEN" will appear in green!\\

\note{
	When you use the color keys inside a {\ttfamily PRINT} statement, a special
	symbol appears in your program listing.  Don't worry --- this is normal!
	The symbol tells the Commander X16 to change colors when the program runs.\\
}

\section{Reverse Mode}

You may have noticed that \key{9} has "RVS ON" and \key{0} has "RVS OFF" on the
keyboard.  Pressing \ctrlkey + \key{9} turns on \emph{reverse mode}, which
swaps the foreground and background colors.  Pressing \ctrlkey + \key{0} turns
reverse mode off.\\

Reverse mode is very useful for creating highlighted text or drawing solid
blocks of color.  Try this:\\

\codeblock{
	10 COLOR 1,6\\
	20 PRINT "NORMAL ";\\
	30 PRINT "\ctrlkey\key{9}REVERSED\ctrlkey\key{0}";\\
	40 PRINT " NORMAL"\\
}

%----------------------------------------------------------------------------------------
%	CHAPTER - Screen and Border Colors
%----------------------------------------------------------------------------------------

\chapter*{Screen and Border Colors}\index{Screen Colors}\index{Border Colors}
\addcontentsline{toc}{chapter}{\protect\numberline{}Screen and Border Colors}

So far we've been changing the color of individual characters.  But what if you
want to change the color of the entire screen or add a colorful border around
it?  The Commander X16 has several ways to do this!\\

\section{Clearing the Screen with Color}

One quick way to fill the screen with a background color is to set the
background color with {\ttfamily COLOR} and then clear the screen.  Remember
from Chapter 1 that you can clear the screen by holding \shiftkey and pressing
\clrhomekey.  You can also clear the screen from a program using {\ttfamily
PRINT} with the clear screen character inside it:\\

\codeblock{
	10 COLOR 1,6\\
	20 PRINT "\shiftkey\clrhomekey"\\
	30 PRINT "BLUE BACKGROUND!"\\
}

When you run this program, the entire screen turns blue and the text appears in
white!\\

\section{Screen Modes with Borders}

The Commander X16 can display a border around the screen, just like the classic
Commodore computers.  To use a screen mode with a border, use the {\ttfamily
SCREEN} command with mode 7 or higher:\\

\codeblock{
	SCREEN 7\\
}

Screen mode 7 gives you a 22 by 23 character display with a nice border around
it.  The border color is determined by the first color in the palette (color
0).\\

\begin{tabular}{|c|l|}
	\hline
	{\bfseries Mode} & {\bfseries Description}\\ \hline
	7 & 22x23 Text with border\\ \hline
	8 & 64x50 Text with border\\ \hline
	9 & 64x25 Text with border\\ \hline
	10 & 32x50 Text with border\\ \hline
	11 & 32x25 Text with border\\ \hline
\end{tabular}

\vspace{16pt}

To go back to the normal 80x60 screen without a border, type:\\

\codeblock{
	SCREEN 0\\
}

%----------------------------------------------------------------------------------------
%	CHAPTER - Screen Locations
%----------------------------------------------------------------------------------------

\chapter*{Screen Locations}\index{Screen Locations}\index{LOCATE}\index{TILE}
\addcontentsline{toc}{chapter}{\protect\numberline{}Screen Locations}

Sometimes you want to put text or graphics at a specific spot on the screen.
The Commander X16 gives you several ways to control exactly where things
appear.\\

\section{The LOCATE Statement}

The {\ttfamily LOCATE} statement moves the cursor to a specific row on the
screen.  The rows are numbered starting from 1 at the top.  After using
{\ttfamily LOCATE}, the next {\ttfamily PRINT} statement will start printing
at that row:\\

\codeblock{
	10 PRINT "\shiftkey\clrhomekey"\\
	20 LOCATE 10\\
	30 PRINT "THIS IS ROW 10"\\
	40 LOCATE 20\\
	50 PRINT "THIS IS ROW 20"\\
}

\section{The TAB Function}

The {\ttfamily TAB} function lets you move to a specific column (horizontal
position) within a {\ttfamily PRINT} statement.  Columns are numbered starting
from 1 on the left:\\

\codeblock{
	10 PRINT "\shiftkey\clrhomekey"\\
	20 PRINT TAB(20);"COLUMN 20"\\
	30 PRINT TAB(40);"COLUMN 40"\\
}

\section{The TILE Statement}

The {\ttfamily TILE} statement is a powerful command that lets you place any
character at any position on the screen using X and Y coordinates.  Unlike
{\ttfamily LOCATE} and {\ttfamily PRINT}, the {\ttfamily TILE} statement uses
coordinates that start from 0, and lets you specify both the column (X) and row
(Y) at the same time:\\

\codeblock{
	10 PRINT "\shiftkey\clrhomekey"\\
	20 TILE 10,5,1 : REM PUT 'A' AT COLUMN 10, ROW 5\\
	30 TILE 20,10,2 : REM PUT 'B' AT COLUMN 20, ROW 10\\
	40 TILE 30,15,3 : REM PUT 'C' AT COLUMN 30, ROW 15\\
}

The third number in the {\ttfamily TILE} statement is the \emph{screen code}
of the character you want to display.  Screen code 1 is the letter A, 2 is B,
and so on.  You can find a complete list of screen codes in the appendix.\\

The {\ttfamily TILE} statement can also set the color of the character by
adding a fourth number.  This number combines the foreground and background
colors into a single value:\\

\codeblock{
	10 PRINT "\shiftkey\clrhomekey"\\
	20 TILE 10,5,1,\$61 : REM 'A' IN WHITE ON BLUE\\
	30 TILE 20,10,2,\$25 : REM 'B' IN GREEN ON RED\\
}

The color value is calculated by putting the background color in the upper 4
bits and the foreground color in the lower 4 bits.  For example, \$61 means
background color 6 (blue) and foreground color 1 (white).\\

\note{

	On the original Commodore 64, programmers used {\ttfamily POKE} commands to
	write characters and colors directly to screen memory.  The Commander X16
	can do this too using {\ttfamily VPOKE} to write to the VERA's video memory,
	but the memory addresses and organization are different from the C64.  The
	good news is that the {\ttfamily COLOR} and {\ttfamily TILE} statements make
	this much easier!  Instead of calculating memory addresses and remembering
	offsets, you can simply tell the X16 \emph{where} you want something and
	\emph{what color} it should be.  The X16 handles all the complicated memory
	details for you.\\

}

%----------------------------------------------------------------------------------------
%	CHAPTER - Random Colors
%----------------------------------------------------------------------------------------

\chapter*{Random Colors}\index{Random Colors}\index{RND}
\addcontentsline{toc}{chapter}{\protect\numberline{}Random Colors}

One of the most fun things you can do with colors is to use \emph{random}
colors!  The {\ttfamily RND} function generates random numbers, and we can use
it to pick random colors for our programs.\\

\section{A Colorful Screen Filler}

Here's a program that fills the screen with randomly colored characters:\\

\codeblock{
	10 COLOR RND(1)*16,RND(1)*16\\
	20 PRINT "*";\\
	30 GOTO 10\\
}

Run this program and watch the screen fill up with colorful asterisks!  Each
asterisk has a random foreground color and a random background color.  Press
\runstopkey to stop the program when you've seen enough.\\

Let's look at how this works.  {\ttfamily RND(1)} gives us a random number
between 0 and 1, like 0.5 or 0.234.  When we multiply it by 16, we get a random
number between 0 and 16.  The {\ttfamily COLOR} statement only uses the whole
number part, so we end up with a random color from 0 to 15 --- exactly the
range we need!\\

\section{Rainbow Text}

Here's a fun program that prints a message in rainbow colors:\\

\codeblock{
	10 PRINT "\shiftkey\clrhomekey"\\
	20 A\$="COMMANDER X16 IS COLORFUL! "\\
	30 FOR I=1 TO LEN(A\$)\\
	40 COLOR (I-1) AND 15\\
	50 PRINT MID\$(A\$,I,1);\\
	60 NEXT I\\
	70 PRINT\\
	80 GOTO 20\\
}

This program prints each letter of the message in a different color, cycling
through all 16 colors.  The {\ttfamily AND 15} makes sure the color number
stays in the valid range of 0-15.\\

\section{Color Cycling}

Here's a mesmerizing program that continuously changes colors:\\

\codeblock{
	10 FOR C=0 TO 15\\
	20 COLOR C,15-C\\
	30 PRINT "COLOR ";C\\
	40 SLEEP 30\\
	50 NEXT C\\
	60 GOTO 10\\
}

This program cycles through all the colors, showing each color number and using
complementary foreground and background colors.  The {\ttfamily SLEEP 30}
command pauses for about half a second so you can see each color.\\

%----------------------------------------------------------------------------------------
%	CHAPTER - Keyboard Graphics
%----------------------------------------------------------------------------------------

\chapter*{Keyboard Graphics}\index{Keyboard Graphics}\index{PETSCII}\index{Graphic Characters}
\addcontentsline{toc}{chapter}{\protect\numberline{}Keyboard Graphics}

The Commander X16 keyboard has many special \emph{graphic characters} that you
can use to draw pictures, borders, and designs.  These characters are part of
the PETSCII character set, named after the classic Commodore PET computer.\\

\section{Typing Graphic Characters}

Look at your keyboard.  Most keys have one or two small symbols on the front
of the key.  These are graphic characters!  To type them:\\

\begin{itemize}
	\item Hold \shiftkey and press a key to type the graphic on the
	      \emph{right} side of the key
	\item Hold \altkey and press a key to type the graphic on the
	      \emph{left} side of the key
\end{itemize}

For example, pressing \shiftkey + \key{S} types a heart symbol, while
\altkey + \key{S} types a corner piece.\\

\section{Drawing Boxes}

One of the most useful things you can do with graphic characters is draw boxes
and borders.  Here are the keys for drawing a simple box:\\

\begin{tabular}{|c|c|l|}
	\hline
	{\bfseries Keys} & {\bfseries Character} & {\bfseries Description}\\ \hline
	\altkey + \key{U} & Rounded top-left corner & Corner piece \\ \hline
	\altkey + \key{I} & Rounded top-right corner & Corner piece \\ \hline
	\altkey + \key{J} & Rounded bottom-left corner & Corner piece \\ \hline
	\altkey + \key{K} & Rounded bottom-right corner & Corner piece \\ \hline
	\shiftkey + \key{-} & Horizontal line & Top/bottom edges \\ \hline
	\shiftkey + \key{|} & Vertical line & Left/right edges \\ \hline
\end{tabular}

\vspace{16pt}

Try drawing a box by typing these characters.  Use the arrow keys to position
your cursor, and use \shiftkey and \altkey with the keys above to draw the
edges and corners.\\

\section{A Box-Drawing Program}

Here's a program that draws a colorful box on the screen:\\

\codeblock{
	10 PRINT "\shiftkey\clrhomekey"\\
	20 COLOR 14,6\\
	30 REM TOP OF BOX\\
	40 PRINT "\altkey\key{U}";\\
	50 FOR I=1 TO 20:PRINT "\shiftkey\key{-}";:NEXT\\
	60 PRINT "\altkey\key{I}"\\
	70 REM SIDES OF BOX\\
	80 FOR J=1 TO 5\\
	90 PRINT "\shiftkey\key{|}";TAB(22);"\shiftkey\key{|}"\\
	100 NEXT J\\
	110 REM BOTTOM OF BOX\\
	120 PRINT "\altkey\key{J}";\\
	130 FOR I=1 TO 20:PRINT "\shiftkey\key{-}";:NEXT\\
	140 PRINT "\altkey\key{K}"\\
}

\note{
	When you type this program, the graphic characters will appear as symbols
	in your listing.  The notation \altkey\key{U} means "hold ALT and press U"
	to type the rounded corner character.\\
}

\section{Making Pictures}

With practice, you can create amazing pictures using graphic characters!  The
secret is to combine different characters and colors.  Here are some ideas to
get you started:\\

\begin{itemize}
	\item Use \shiftkey + \key{Q} and \shiftkey + \key{W} for ball shapes
	\item Use \shiftkey + \key{A} and \shiftkey + \key{S} for card suit symbols
	\item Use \altkey + \key{A}, \key{S}, \key{Z}, \key{X} for quarter-block
	      characters that can make smooth diagonal lines
	\item Combine reverse mode with spaces to make solid color blocks
\end{itemize}

\tryit{
	Try creating your own picture using graphic characters!  Start simple ---
	maybe a house, a car, or a smiley face.  Use {\ttfamily COLOR} to add colors
	to different parts of your picture.\\
}

\section{The Character Set}

The Commander X16 has 256 different characters, including all the letters,
numbers, punctuation, and graphic symbols.  You can see all of them using this
program:\\

\codeblock{
	10 PRINT "\shiftkey\clrhomekey"\\
	20 FOR I=0 TO 255\\
	30 TILE I AND 15, I/16, I\\
	40 NEXT I\\
}

This uses the {\ttfamily TILE} statement to display all 256 characters in a
16 by 16 grid.  The character codes go from 0 to 255.  You can find a complete
chart of all characters and their codes in the appendix.\\

%----------------------------------------------------------------------------------------
%	CHAPTER - Combining Color and Sound
%----------------------------------------------------------------------------------------

\chapter*{Combining Color and Sound}\index{Color and Sound}
\addcontentsline{toc}{chapter}{\protect\numberline{}Combining Color and Sound}

Colors and sounds go great together!  Here's a simple program that makes random
colors and plays random notes at the same time:\\

\codeblock{
	10 COLOR RND(1)*16,RND(1)*16\\
	20 PRINT "\shiftkey\key{Q}";\\
	30 N\$="CDEFGAB"\\
	40 PSGPLAY 0,"O3"+MID\$(N\$,RND(1)*7+1,1)\\
	50 SLEEP 10\\
	60 GOTO 10\\
}

This program fills the screen with colorful circles while playing random
musical notes.  You'll learn more about making music in Chapter 5 --- for now,
just enjoy the colorful noise!\\

\tip{Making Your Own Light Show}{

	Try modifying the program above to create your own audio-visual experience!
	Change the graphic character to something else, adjust the {\ttfamily SLEEP}
	time to speed up or slow down the show, or use {\ttfamily TILE} to place
	characters in specific patterns while the music plays.\\

}

\@openrighttrue\makeatother
