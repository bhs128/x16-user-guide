%----------------------------------------------------------------------------------------
%	PART - Setup
%----------------------------------------------------------------------------------------

\makeatletter\@openrightfalse
\part{Unpacking and Connecting}

%----------------------------------------------------------------------------------------
%	CHAPTER - Unpacking the Commander X16
%----------------------------------------------------------------------------------------

\chapter*{Unpacking the Commander X16}\index{Setup}\index{Connection}
\addcontentsline{toc}{chapter}{\protect\numberline{}Unpacking the Commander X16}

Welcome to computing!  The following step-by-step instructions show you how to
unpack your Commander X16, connect it to your display, and make sure it's
working properly.\\

Let's begin by taking a quick look at the Commander X16.

%----------------------------------------------------------------------------------------
%	SECTION - What's in the Box
%----------------------------------------------------------------------------------------

\section*{What's in the Box}\index{Setup!contents}

Check the contents of your Commander X16 package.  You should find the
following items:

\begin{itemize}
	\item Commander X16 Motherboard
	\item SD Card (pre-loaded with software)
	\item Quick Setup Guide pamphlet
\end{itemize}

If you purchased the 2MB RAM / User Port Upgrade kit, you will also have those
components included. 

\note{
	The core Commander X16 does \textbf{not} include a power supply, keyboard, mouse,
	or display.  These need to be purchased separately or you will need to provide your own.
}

%----------------------------------------------------------------------------------------
%	SECTION - What You'll Need
%----------------------------------------------------------------------------------------

\section*{What You'll Need}\index{Setup!requirements}

Before you can use your Commander X16, you will need the following items:

\begin{itemize}
	\item \textbf{Power Supply}: The X16 draws only a few watts of power, so an
		ATX picoPSU (80--160W) with a 12VDC wall adapter is recommended.  You
		can also use a standard full-size ATX power supply if that's what you
		have available.
	\item \textbf{Display}: A VGA monitor works best.  You can also use S-Video
		or Composite video connections for older CRT or TV displays.
	\item \textbf{Video Cable}: A corresponding video cable to connect the X16 to your
		display (often included with monitors).
	\item \textbf{Keyboard}: A PS/2 keyboard.  The official Commander X16
		keyboard is highly recommended!
	\item \textbf{Speakers or Headphones}: The X16 has a stereo audio jack for
		sound output.
\end{itemize}

\tip{The Official Keyboard}{
	The official Commander X16 keyboard includes special keys like
	\runstopkey{} and function keys that match the commands printed in
	this manual.  It also includes graphic symbols printed on the key fronts,
	just like the classic Commodore keyboards!
}

\note{
	Some USB keyboards will \textbf{not} work with a USB-to-PS/2 adapter because
	they lack support for the legacy PS/2 protocol.  For the best experience,
	use a keyboard with a native PS/2 connector.
}

%----------------------------------------------------------------------------------------
%	SECTION - About the Case
%----------------------------------------------------------------------------------------

\section*{About the Case}\index{Setup!case}

The X16 motherboard uses the standard Micro-ATX form factor, which has been the
industry standard since 1995.  This means you can mount it in any Micro-ATX-compatible
case---either a ``desktop'' (lower profile) or ``tower'' (upright) style.\\

If you don't yet have a case, the X16 board can safely sit atop the thin padding
it came with.  Just avoid flexing or twisting the board---keep it flat!\\

If you have the official Lazer3D case, assembly takes about 20--40 minutes and
requires only a regular Philips head screwdriver.  The case comes pre-fitted
with the 9 stand-off screws needed to mount the motherboard.

\note{
	Remember to ground yourself by touching a grounded metal object (try a light switch cover plate screw) before handling electronic equipment!
}

%----------------------------------------------------------------------------------------
%	CHAPTER - Connecting Your X16
%----------------------------------------------------------------------------------------

\chapter*{Connecting Your X16}\index{Setup!connecting}
\addcontentsline{toc}{chapter}{\protect\numberline{}Connecting Your X16}

Position your Commander X16 and display so that you can use the keyboard
comfortably while viewing the screen---ideally on a desk or tabletop.  Now
follow these steps:

\begin{enumerate}
	\item \textbf{Insert the SD card} into the SD card slot on the back of the
		X16.  Push the card in gently until it stops (note it will not click).

	\item \textbf{Connect your keyboard} to the \textbf{lower} PS/2 port on
		the back of the X16.  The keyboard port is the lower one closer to the edge
		of the board; the upper port is for the mouse.

	\item \textbf{Connect the video cable} from your display to one of the
		video ports on the back of the X16.  The X16 supports three video
		output options:
		\begin{itemize}
			\item \textbf{VGA} (DB15): Best quality, easiest to adapt to modern
				displays.  Modern VGA cables typically have a blue connector.
			\item \textbf{S-Video}: Better quality than Composite.  Take care
				not to bend the 4 pins within the connector.
			\item \textbf{Composite} (RCA): Yellow connector, easiest to
				connect since orientation doesn't matter.
		\end{itemize}

	\item \textbf{Connect speakers or headphones} to the stereo audio jack,
		located next to the keyboard and mouse connectors.

	\item \textbf{Connect the ATX power supply} to the 24-pin (or 20-pin) ATX
		power connector on the motherboard.  Make sure the connector is fully
		seated---it should only fit one way.

	\item \textbf{Turn on your display} and make sure it is set to the correct
		input or source.

	\item \textbf{Turn on the Commander X16} by pressing the red power button
		on the motherboard (or use the front panel power button if your case
		has one connected).
\end{enumerate}

\note{
	By default, the X16 outputs video in VGA mode.  If you are using S-Video or
	Composite and see no picture, press \shiftkey+\widekey{40/80 DISP} to cycle
	through the video output modes.  You will hear a tone when the mode changes.
	(On a PC keyboard, this key is labeled Scroll Lock.)
}

\tip{Using HDMI Displays}{
	If your display only has HDMI input, you can use a ``VGA to HDMI'' adapter.
	Look for one that is \textit{active} (powered by USB) and includes audio
	support.  The USB power for the adapter should come from a wall adapter,
	not the X16.
}

%----------------------------------------------------------------------------------------
%	SECTION - The Startup Screen
%----------------------------------------------------------------------------------------

\section*{The Startup Screen}\index{Setup!startup screen}

If everything is connected properly, you should see the Commander X16 startup
screen appear on your display after a few seconds:

\begin{center}
	\screenbox{3in}{2in}{
		**** COMMANDER X16 BASIC V2 ****\\
		512K HIGH RAM - ROM VER Vxx\\
		38655 BASIC BYTES FREE\\

		READY.\\
		\cursor
	}
\end{center}

The blinking square is called the \textbf{cursor}.  It shows you where the next
character you type will appear.  When you see the word {\ttfamily READY.} and
the blinking cursor, the Commander X16 is waiting for you to type something!\\

\textbf{Congratulations!}  Your Commander X16 is now ready to use.\\[1em]

Before moving on, type {\ttfamily MENU} and press \returnkey{} to open the
system menu.  Select \textbf{CONTROL PANEL} to review your video output
settings and set the current date and time.  You can save these settings so
they persist when the X16 is powered off.\\[1em]

If everything is working, you're all set!  Turn to Chapter 2 to begin your
computing adventure.  But if you're having trouble getting your X16 to start
up, continue reading for some troubleshooting tips.

%----------------------------------------------------------------------------------------
%	CHAPTER - Troubleshooting
%----------------------------------------------------------------------------------------

\chapter*{Troubleshooting}\index{Setup!troubleshooting}\index{Setup!accessories}
\addcontentsline{toc}{chapter}{\protect\numberline{}Troubleshooting}

%----------------------------------------------------------------------------------------
%	SECTION - Troubleshooting
%----------------------------------------------------------------------------------------

\section*{Troubleshooting}

If you don't see the startup screen, try these solutions:

\begin{description}
	\item[No picture, no power light:] Check that the ATX power connector is fully seated and the PSU switch is on.
	\item[No picture, power light on:] Check your video cable. Press \shiftkey+\widekey{40/80 DISP} to cycle video modes.
	\item[No keyboard response:] Make sure your PS/2 keyboard is in the \textbf{lower} port (the upper port is for mice only).
	\item[No sound:] Connect speakers or headphones to the audio jack. USB speakers are not supported.
	\item[SD card not recognized:] Reinsert the card firmly. Use an 8GB or larger card formatted as FAT32.
\end{description}

We hope these tips have helped get you up and running!  If you're still having
issues, the Commander X16 has a vibrant community of fellow enthusiasts who
would be happy to help.  Check out these online resources for more support:

\begin{itemize}
	\item \textbf{Official Website}: \url{https://www.commanderx16.com}
	\item \textbf{Community Forum}: \url{https://cx16forum.com}
	\item \textbf{Discord}: \url{https://discord.gg/nS2PqEC}
	\item \textbf{Facebook Group}: \url{https://www.facebook.com/groups/CommanderX16}
\end{itemize}

Now that your Commander X16 is set up and running, you're ready to start
exploring!  Turn the page to begin Chapter 2: Getting to Know Your Commander X16.


