%----------------------------------------------------------------------------------------
%	PART - Animation
%----------------------------------------------------------------------------------------

\makeatletter\@openrightfalse
\part{Animation}

\chaptertypein{
	\keybackgroundcolor{gray}
	\keytextcolor{black}
	10 PRINT "\shiftkey\clrhomekey"\\
	20 FOR X=1 TO 38\\
	30 PRINT "\clrhomekey";\\
	40 PRINT TAB(X);"*"\\
	50 FOR D=1 TO 50:NEXT D\\
	60 NEXT X\\
	70 GOTO 20
}

%----------------------------------------------------------------------------------------
%	CHAPTER - Animation
%----------------------------------------------------------------------------------------

\chapter*{Animation}\index{Animation}
\addcontentsline{toc}{chapter}{\protect\numberline{}Animation}

% TODO: This chapter needs content covering (VIC-20 style):
% - Flying birds example
% - Bouncing ball
% - Cursor control for animation
% - Animating with POKEs and PEEKs
% - Simple game animations

\section{Introduction}

% Placeholder content - replace with actual content
One of the most exciting things you can do with your Commander X16 is create
animations! In this chapter, you'll learn how to make objects move across the
screen, bounce around, and come to life.

\section{Your First Animation}

% TODO: Simple moving character animation

\section{The Bouncing Ball}

% TODO: Classic bouncing ball program

\section{Cursor Control}

% TODO: Using cursor movement for animation

\section{Animation with PEEK and POKE}

% TODO: Direct screen memory animation

\section{Flying Objects}

% TODO: Bird or spaceship animation example

\@openrighttrue\makeatother
