%----------------------------------------------------------------------------------------
%	APPENDIX - Sample Programs
%----------------------------------------------------------------------------------------

\chapter*{Sample Programs}\index{Sample Programs}
\addcontentsline{toc}{chapter}{\protect\numberline{}Sample Programs}

\section*{Introduction}

This appendix contains a collection of programs for you to type in and enjoy.
Each program demonstrates different capabilities of the Commander X16.  Feel
free to modify these programs and experiment with changes to see what happens!

These programs range from simple games to useful utilities.  They are designed
to be typed in exactly as shown, but don't be afraid to experiment once you have
them working.  Try changing numbers, colors, or messages to see what happens.
That's how programmers learn!

\tip{Saving Your Work}{
	After typing in a long program, save it to your SD card before running it!
	Use {\ttfamily SAVE "FILENAME"} to save your work.  That way, if something
	goes wrong, you won't have to type it all in again.
}

%----------------------------------------------------------------------------------------
%	PROGRAM 1 - HI-LO GUESSING GAME
%----------------------------------------------------------------------------------------

\section*{Game: Hi-Lo}
\index{Sample Programs!Hi-Lo}

This classic guessing game challenges you to find the computer's secret number.
The X16 picks a random number between 1 and 100, and you try to guess it.  After
each guess, the computer tells you if your guess was too high or too low.  See
how few guesses you can make!

\codeblock{
	100 REM ** HI-LO GUESSING GAME **\\
	110 PRINT "\shiftkey\clrhomekey"\\
	120 PRINT "** HI-LO **"\\
	130 PRINT "I'M THINKING OF A NUMBER"\\
	140 PRINT "BETWEEN 1 AND 100."\\
	150 PRINT\\
	160 N = INT(RND(1)*100)+1\\
	170 G = 0\\
	180 PRINT "YOUR GUESS";\\
	190 INPUT W\\
	200 G = G + 1\\
	210 IF W < N THEN PRINT "TOO LOW!": GOTO 180\\
	220 IF W > N THEN PRINT "TOO HIGH!": GOTO 180\\
	230 PRINT\\
	240 PRINT "YOU GOT IT IN"; G; "GUESSES!"\\
	250 PRINT\\
	260 IF G <= 7 THEN PRINT "EXCELLENT!"\\
	270 IF G > 7 AND G <= 10 THEN PRINT "GOOD JOB!"\\
	280 IF G > 10 THEN PRINT "YOU CAN DO BETTER!"\\
	290 PRINT\\
	300 PRINT "PLAY AGAIN (Y/N)";\\
	310 GET A\$: IF A\$ = "" THEN 310\\
	320 IF A\$ = "Y" THEN 110\\
	330 IF A\$ <> "N" THEN 310\\
	340 PRINT\\
	350 PRINT "THANKS FOR PLAYING!"\\
}

\textbf{How It Works:}

\begin{itemize}
	\item Line 160 picks a random number between 1 and 100 and stores it in
		{\ttfamily N}.
	\item Line 170 initializes the guess counter {\ttfamily G} to zero.
	\item Lines 180--220 form the main game loop: get a guess, count it, and
		check if it's too low, too high, or correct.
	\item Lines 260--280 rate your performance based on number of guesses.
	\item Lines 300--330 use {\ttfamily GET} to ask if you want to play again.
\end{itemize}

\tryit{
	Can you modify the program to let the player choose the range?  Try adding
	an {\ttfamily INPUT} at the beginning to ask for the maximum number!
}

%----------------------------------------------------------------------------------------
%	PROGRAM 2 - REACTION TIMER
%----------------------------------------------------------------------------------------

\section*{Game: Reaction Timer}
\index{Sample Programs!Reaction Timer}

How fast are your reflexes?  This program tests your reaction time by displaying
a message at a random moment and measuring how quickly you press a key.  The X16
uses its built-in timer ({\ttfamily TI}) to measure time in ``jiffies''---60ths
of a second.

\codeblock{
	100 REM ** REACTION TIMER **\\
	110 PRINT "\shiftkey\clrhomekey"\\
	120 PRINT "** REACTION TIMER **"\\
	130 PRINT\\
	140 PRINT "WHEN YOU SEE 'GO!', PRESS"\\
	150 PRINT "ANY KEY AS FAST AS YOU CAN!"\\
	160 PRINT\\
	170 PRINT "PRESS ANY KEY TO START..."\\
	180 GET A\$: IF A\$ = "" THEN 180\\
	190 PRINT "\shiftkey\clrhomekey"\\
	200 PRINT "GET READY..."\\
	210 W = INT(RND(1)*180)+60\\
	220 FOR D = 1 TO W: NEXT D\\
	230 PRINT\\
	240 PRINT "     **** GO! ****"\\
	250 TI\$ = "000000"\\
	260 GET A\$: IF A\$ = "" THEN 260\\
	270 T = TI\\
	280 PRINT\\
	290 MS = INT(T * 16.67)\\
	300 PRINT "YOUR TIME:"; MS; "MILLISECONDS"\\
	310 PRINT\\
	320 IF MS < 200 THEN PRINT "INCREDIBLE!"\\
	330 IF MS >= 200 AND MS < 300 THEN PRINT "EXCELLENT!"\\
	340 IF MS >= 300 AND MS < 400 THEN PRINT "GOOD!"\\
	350 IF MS >= 400 THEN PRINT "KEEP PRACTICING!"\\
	360 PRINT\\
	370 PRINT "TRY AGAIN (Y/N)";\\
	380 GET A\$: IF A\$ = "" THEN 380\\
	390 IF A\$ = "Y" THEN 110\\
	400 PRINT\\
}

\textbf{How It Works:}

\begin{itemize}
	\item Line 210 picks a random delay (60--240 jiffies, or 1--4 seconds) so
		you can't predict when ``GO!'' will appear.
	\item Line 250 resets the system timer by setting {\ttfamily TI\$} to all
		zeros.
	\item Line 270 captures the timer value immediately after you press a key.
	\item Line 290 converts jiffies to milliseconds (each jiffy is about 16.67
		milliseconds).
\end{itemize}

\note{
	The {\ttfamily TI} variable is a special system variable that counts up
	automatically, 60 times per second.  Setting {\ttfamily TI\$} to
	{\ttfamily "000000"} resets it to zero.  This is a handy way to measure
	elapsed time in your programs!
}

%----------------------------------------------------------------------------------------
%	PROGRAM 3 - DICE ROLLER
%----------------------------------------------------------------------------------------

\section*{Game: Dice Roller}
\index{Sample Programs!Dice Roller}

This program simulates rolling dice with a fun animated display.  You can
choose how many dice to roll (1 to 5), and watch them ``tumble'' before showing
the final result.  Great for board games!

\codeblock{
	100 REM ** DICE ROLLER **\\
	110 PRINT "\shiftkey\clrhomekey"\\
	120 PRINT "** DICE ROLLER **"\\
	130 PRINT\\
	140 PRINT "HOW MANY DICE (1-5)";\\
	150 INPUT ND\\
	160 IF ND < 1 OR ND > 5 THEN 140\\
	170 PRINT\\
	180 PRINT "PRESS SPACE TO ROLL..."\\
	190 GET A\$: IF A\$ <> " " THEN 190\\
	200 PRINT\\
	210 REM ANIMATE THE ROLL\\
	220 FOR R = 1 TO 10\\
	230 PRINT "\clrhomekey";\\
	240 FOR I = 1 TO 5: PRINT: NEXT I\\
	250 PRINT "ROLLING: ";\\
	260 FOR I = 1 TO ND\\
	270 PRINT INT(RND(1)*6)+1; " ";\\
	280 NEXT I\\
	290 FOR D = 1 TO 30: NEXT D\\
	300 NEXT R\\
	310 REM SHOW FINAL RESULT\\
	320 PRINT "\clrhomekey";\\
	330 FOR I = 1 TO 5: PRINT: NEXT I\\
	340 PRINT "RESULT:  ";\\
	350 T = 0\\
	360 FOR I = 1 TO ND\\
	370 V = INT(RND(1)*6)+1\\
	380 PRINT V; " ";\\
	390 T = T + V\\
	400 NEXT I\\
	410 PRINT\\
	420 PRINT\\
	430 PRINT "TOTAL:"; T\\
	440 PRINT\\
	450 PRINT "ROLL AGAIN (Y/N)";\\
	460 GET A\$: IF A\$ = "" THEN 460\\
	470 IF A\$ = "Y" THEN 170\\
	480 IF A\$ <> "N" THEN 460\\
}

\textbf{How It Works:}

\begin{itemize}
	\item Lines 220--300 create the ``tumbling'' animation by rapidly
		displaying random numbers.
	\item Line 290 is a short delay loop that makes the animation visible.
	\item Lines 350--400 generate and display the final dice values while
		adding them up in {\ttfamily T}.
\end{itemize}

%----------------------------------------------------------------------------------------
%	PROGRAM 4 - MINI CALCULATOR
%----------------------------------------------------------------------------------------

\section*{Utility: Mini Calculator}
\index{Sample Programs!Calculator}

This handy calculator lets you perform basic math operations.  Enter two numbers
and choose an operation, and the X16 gives you the answer.  It even handles
errors gracefully!

\codeblock{
	100 REM ** MINI CALCULATOR **\\
	110 PRINT "\shiftkey\clrhomekey"\\
	120 PRINT "** MINI CALCULATOR **"\\
	130 PRINT\\
	140 PRINT "ENTER FIRST NUMBER";\\
	150 INPUT A\\
	160 PRINT "ENTER SECOND NUMBER";\\
	170 INPUT B\\
	180 PRINT\\
	190 PRINT "CHOOSE OPERATION:"\\
	200 PRINT "1) ADD"\\
	210 PRINT "2) SUBTRACT"\\
	220 PRINT "3) MULTIPLY"\\
	230 PRINT "4) DIVIDE"\\
	240 PRINT "5) POWER"\\
	250 PRINT\\
	260 PRINT "YOUR CHOICE";\\
	270 INPUT C\\
	280 PRINT\\
	290 IF C = 1 THEN PRINT A; "+"; B; "="; A + B\\
	300 IF C = 2 THEN PRINT A; "-"; B; "="; A - B\\
	310 IF C = 3 THEN PRINT A; "*"; B; "="; A * B\\
	320 IF C = 4 AND B <> 0 THEN PRINT A; "/"; B; "="; A / B\\
	330 IF C = 4 AND B = 0 THEN PRINT "ERROR: DIVIDE BY ZERO!"\\
	340 IF C = 5 THEN PRINT A; "\^{}"; B; "="; A \^{} B\\
	350 PRINT\\
	360 PRINT "ANOTHER CALCULATION (Y/N)";\\
	370 GET A\$: IF A\$ = "" THEN 370\\
	380 IF A\$ = "Y" THEN 110\\
	390 IF A\$ <> "N" THEN 370\\
	400 PRINT\\
	410 PRINT "GOODBYE!"\\
}

\textbf{How It Works:}

\begin{itemize}
	\item Lines 140--170 get the two numbers to work with.
	\item Lines 190--270 display a menu and get the user's choice.
	\item Lines 290--340 perform the selected operation and display the result.
	\item Line 330 checks for division by zero before it happens!
\end{itemize}

\tryit{
	Can you add more operations?  Try adding square root ({\ttfamily SQR(A)}),
	absolute value ({\ttfamily ABS(A)}), or modulo (remainder after division).
}

%----------------------------------------------------------------------------------------
%	PROGRAM 5 - PATTERN MAKER
%----------------------------------------------------------------------------------------

\section*{Graphics: Pattern Maker}
\index{Sample Programs!Pattern Maker}

This program creates colorful patterns on the screen using PETSCII graphics
characters.  Watch as the X16 fills the screen with an ever-changing display of
shapes and colors!

\codeblock{
	100 REM ** PATTERN MAKER **\\
	110 PRINT "\shiftkey\clrhomekey"\\
	120 REM PETSCII SHAPE CHARACTERS\\
	130 DIM S\$(5)\\
	140 S\$(1) = "*"\\
	150 S\$(2) = "+"\\
	160 S\$(3) = "O"\\
	170 S\$(4) = "X"\\
	180 S\$(5) = "@"\\
	190 REM MAIN LOOP\\
	200 C = INT(RND(1)*15)+1\\
	210 PRINT CHR\$(28+C);\\
	220 X = INT(RND(1)*40)\\
	230 Y = INT(RND(1)*30)\\
	240 P = INT(RND(1)*5)+1\\
	250 PRINT CHR\$(19);\\
	260 FOR I = 1 TO Y: PRINT: NEXT I\\
	270 PRINT TAB(X); S\$(P);\\
	280 GET A\$: IF A\$ <> "" THEN 300\\
	290 GOTO 200\\
	300 PRINT "\shiftkey\clrhomekey"\\
}

\textbf{How It Works:}

\begin{itemize}
	\item Lines 130--180 set up an array of different characters to display.
	\item Line 200 picks a random color (1--15).
	\item Line 210 uses {\ttfamily CHR\$(28+C)} to change the text color.
	\item Lines 220--230 pick a random screen position.
	\item Line 250 uses {\ttfamily CHR\$(19)} to move the cursor home without
		clearing the screen.
	\item Line 280 checks if a key was pressed to exit the program.
\end{itemize}

\note{
	The color codes for {\ttfamily CHR\$} range from 29 to 43.  By adding a
	random number (1--15) to 28, we get a random color each time!
}

%----------------------------------------------------------------------------------------
%	PROGRAM 6 - SIMPLE MELODY PLAYER
%----------------------------------------------------------------------------------------

\section*{Sound: Simple Melody Player}
\index{Sample Programs!Melody Player}

This program plays a simple melody using the X16's FM sound chip.  It
demonstrates how to use the {\ttfamily FMPLAY} command to create music with
note strings.  The program plays ``Twinkle, Twinkle, Little Star''!

\codeblock{
	100 REM ** MELODY PLAYER **\\
	110 PRINT "\shiftkey\clrhomekey"\\
	120 PRINT "** MELODY PLAYER **"\\
	130 PRINT\\
	140 PRINT "TWINKLE TWINKLE LITTLE STAR"\\
	150 PRINT\\
	160 REM INITIALIZE FM CHIP\\
	170 FMINIT\\
	180 FMINST 0, 16: REM ORGAN SOUND\\
	190 REM PLAY THE MELODY\\
	200 PRINT "PLAYING..."\\
	210 FMPLAY 0, "T120 L4"\\
	220 FMPLAY 0, "C C G G A A G2"\\
	230 FMPLAY 0, "F F E E D D C2"\\
	240 FMPLAY 0, "G G F F E E D2"\\
	250 FMPLAY 0, "G G F F E E D2"\\
	260 FMPLAY 0, "C C G G A A G2"\\
	270 FMPLAY 0, "F F E E D D C2"\\
	280 PRINT\\
	290 PRINT "DONE!"\\
	300 PRINT\\
	310 PRINT "PLAY AGAIN (Y/N)";\\
	320 GET A\$: IF A\$ = "" THEN 320\\
	330 IF A\$ = "Y" THEN 110\\
}

\textbf{How It Works:}

\begin{itemize}
	\item Line 170 initializes the FM sound chip.
	\item Line 180 loads instrument 16 (organ) into channel 0.
	\item Line 210 sets the tempo (120 beats per minute) and default note
		length (quarter notes).
	\item Lines 220--270 play the melody using note letters.  The number after
		a note (like {\ttfamily G2}) makes it a half note (twice as long).
\end{itemize}

\tryit{
	Try changing instrument 16 to other numbers (0--47) to hear different
	sounds!  You can also change the tempo by adjusting the number after
	{\ttfamily T} in line 210.
}

%----------------------------------------------------------------------------------------
%	PROGRAM 7 - NUMBER MEMORY GAME
%----------------------------------------------------------------------------------------

\section*{Game: Number Memory}
\index{Sample Programs!Number Memory}

Test your memory with this challenging game!  The computer displays a sequence
of numbers, then hides them.  Can you remember what they were and type them back
in the correct order?  The sequences get longer as you succeed!

\codeblock{
	100 REM ** NUMBER MEMORY **\\
	110 PRINT "\shiftkey\clrhomekey"\\
	120 PRINT "** NUMBER MEMORY GAME **"\\
	130 PRINT\\
	140 PRINT "MEMORIZE THE NUMBERS,"\\
	150 PRINT "THEN TYPE THEM BACK!"\\
	160 PRINT\\
	170 DIM N(10)\\
	180 LV = 3: REM STARTING LENGTH\\
	190 SC = 0: REM SCORE\\
	200 REM GENERATE SEQUENCE\\
	210 PRINT "LEVEL"; LV - 2\\
	220 PRINT "MEMORIZE: ";\\
	230 FOR I = 1 TO LV\\
	240 N(I) = INT(RND(1)*10)\\
	250 PRINT N(I);\\
	260 NEXT I\\
	270 PRINT\\
	280 REM PAUSE TO MEMORIZE\\
	290 FOR D = 1 TO LV * 60: NEXT D\\
	300 PRINT "\shiftkey\clrhomekey"\\
	310 PRINT "NOW TYPE THEM BACK:"\\
	320 PRINT\\
	330 REM CHECK ANSWERS\\
	340 FOR I = 1 TO LV\\
	350 PRINT "NUMBER"; I;\\
	360 INPUT G\\
	370 IF G <> N(I) THEN 430\\
	380 NEXT I\\
	390 PRINT\\
	400 PRINT "CORRECT! WELL DONE!"\\
	410 SC = SC + LV\\
	420 LV = LV + 1: IF LV > 10 THEN LV = 10\\
	425 PRINT "SCORE:"; SC\\
	426 PRINT\\
	427 PRINT "PRESS ANY KEY..."\\
	428 GET A\$: IF A\$ = "" THEN 428\\
	429 GOTO 200\\
	430 PRINT\\
	440 PRINT "WRONG! THE SEQUENCE WAS:"\\
	450 FOR I = 1 TO LV\\
	460 PRINT N(I);\\
	470 NEXT I\\
	480 PRINT\\
	490 PRINT "FINAL SCORE:"; SC\\
	500 PRINT\\
	510 PRINT "PLAY AGAIN (Y/N)";\\
	520 GET A\$: IF A\$ = "" THEN 520\\
	530 IF A\$ = "Y" THEN 110\\
}

\textbf{How It Works:}

\begin{itemize}
	\item Line 170 creates an array to store the number sequence.
	\item Line 180 starts with 3 numbers to memorize.
	\item Lines 230--260 generate and display random numbers.
	\item Line 290 pauses for a time proportional to the sequence length.
	\item Lines 340--380 check each answer against the stored sequence.
	\item Line 420 increases difficulty (up to 10 numbers maximum).
\end{itemize}

\tip{Memory Tip}{
	Try grouping the numbers in your mind.  For example, ``3 7 2 5'' might be
	easier to remember as ``37'' and ``25'' --- two two-digit numbers instead
	of four single digits!
}
