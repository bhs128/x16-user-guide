%----------------------------------------------------------------------------------------
%	PART - Introduction to Programming
%----------------------------------------------------------------------------------------

\makeatletter\@openrightfalse
\part{Introduction to Programming}

\chaptertypein{
	\keybackgroundcolor{gray}
	\keytextcolor{black}
	10 REM GUESSING GAME\\
	20 N = INT(RND(1)*10)+1\\
	30 PRINT "GUESS 1-10:"\\
	40 INPUT G\\
	50 IF G = N THEN GOTO 80\\
	60 PRINT "TRY AGAIN!"\\
	70 GOTO 30\\
	80 PRINT "YOU GOT IT!"
}

\begin{tikzpicture}
	\hyphenpenalty=10000
	\bubble{2.5in}{2.6in}{5.0}{6.6}{
		The X16 picks a secret number between 1 and 10.}
	\bubble{2.5in}{1.9in}{4.2}{5.6}{
		You type your guess and the X16 remembers it as G.}
	\bubble{2.5in}{1.2in}{5.3}{3.5}{
		If your guess equals the secret number, you win!}
\end{tikzpicture}

\newpage
\partheading{Introduction to Programming}

Now that you've learned the basics of using your Commander X16---printing to
the screen, using variables, and getting input from the keyboard---it's time to
put it all together and write some real programs!

In this chapter, you'll learn some important programming concepts that will let
you create games, solve problems, and have fun with your X16.  We'll explore
how to make your programs unpredictable with random numbers, how to make
decisions with {\ttfamily IF...THEN}, and how to repeat actions with {\ttfamily
FOR...NEXT} loops.

By the end of this chapter, you'll have all the tools you need to write your
own games and useful programs!

%----------------------------------------------------------------------------------------
%	CHAPTER - Random Numbers
%----------------------------------------------------------------------------------------

\chapter*{Random Numbers}\index{RND}\index{Random Numbers}
\addcontentsline{toc}{chapter}{\protect\numberline{}Random Numbers}

Many times when you're writing a game or a fun program, you need to generate a
random number.  For example, if you're simulating the throw of dice, you need
the computer to pick a number between 1 and 6---and you don't want to know what
it will be ahead of time!

The Commander X16 has a built-in function called {\ttfamily RND} that generates
random numbers for you.  To see what {\ttfamily RND} does, try this short
program:

\codeblock{
	NEW\\\\
	10 FOR X = 1 TO 10\\
	20 PRINT RND(1),\\
	30 NEXT X\\
}

\reminder{
	The comma at the end of line 20 makes the numbers print across the screen
	in columns instead of down the screen in a single column.
}

After running the program, you will see a display something like this:

\begin{center}
\screenbox{3.5in}{1.5in}{
	.789280697\hspace{1em}.664673958\\
	.256373663\hspace{1em}.0123442287\\
	.682952381\hspace{1em}.390587279\\
	.402343724\hspace{1em}.879300926\\
	.158209063\hspace{1em}.245596701\\
}
\end{center}

Your numbers won't match these---and that's exactly the point!  If they did
match, they wouldn't be very random, would they?

Try running the program a few more times.  You'll notice that the results are
always different, but some things stay the same:

\begin{itemize}
	\item The results are always between 0 and 1 (but never exactly 0 or 1)
	\item The numbers have lots of decimal places
\end{itemize}

This is a problem if we want to simulate rolling dice, since we need whole
numbers between 1 and 6, not fractions!

\section{Getting Useful Random Numbers}

Let's fix this step by step.  First, let's make the numbers bigger by
multiplying.  Replace line 20 with:

\codeblock{
	20 PRINT 6*RND(1),\\
}

Now run the program again.  You should see numbers like:

\begin{center}
\screenbox{3.5in}{1.5in}{
	3.60563664\hspace{1em}4.53660853\\
	5.47238963\hspace{1em}1.40850227\\
	3.19265054\hspace{1em}4.39547668\\
	3.16331095\hspace{1em}5.50620749\\
	2.32527884\hspace{1em}4.17090293\\
}
\end{center}

That's better---the numbers are larger now!  But we still have those pesky
decimal places.  To get rid of them, we can use another function called
{\ttfamily INT}.  The {\ttfamily INT} function takes any number and gives you
just the whole number part (the {\em integer}).  Replace line 20 again:

\codeblock{
	20 PRINT INT(6*RND(1)),\\
}

Run the program and you should see something like:

\begin{center}
\screenbox{3.5in}{1in}{
	2\hspace{2em}3\hspace{2em}1\hspace{2em}0\\
	2\hspace{2em}4\hspace{2em}5\hspace{2em}5\\
	0\hspace{2em}1\\
}
\end{center}

We're getting closer!  Now we have whole numbers.  But look carefully---the
numbers range from 0 to 5, not 1 to 6.  That's because {\ttfamily INT} always
rounds {\em down}, and {\ttfamily RND(1)} never quite reaches 1.

The final step is simple: just add 1!

\codeblock{
	20 PRINT INT(6*RND(1))+1,\\
}

Now you have a perfect dice simulator!  The numbers will always be 1, 2, 3, 4,
5, or 6---just like a real die.

\section{The Random Number Formula}

You can use this same technique to generate random numbers in any range you
want.  Just multiply {\ttfamily RND(1)} by the highest number you want, use
{\ttfamily INT} to get a whole number, and add 1.

For example, to generate random numbers between 1 and 25:

\codeblock{
	20 PRINT INT(25*RND(1))+1,\\
}

The general formula for generating random numbers in a range is:

\begin{center}
{\ttfamily NUMBER = INT(UPPER*RND(1))+1}
\end{center}

Where {\ttfamily UPPER} is the highest number you want.

\tryit{
	Try changing the program to simulate different things:
	\begin{itemize}
		\item A coin flip (1 or 2, for heads or tails)
		\item Drawing a card (1 to 52)
		\item Picking a lottery number (1 to 100)
	\end{itemize}
}

%----------------------------------------------------------------------------------------
%	CHAPTER - Making Decisions with IF...THEN
%----------------------------------------------------------------------------------------

\chapter*{Making Decisions with IF...THEN}\index{IF...THEN}\index{Decisions}
\addcontentsline{toc}{chapter}{\protect\numberline{}Making Decisions with IF...THEN}

So far, your programs have run straight through from beginning to end, or
looped forever with {\ttfamily GOTO}.  But what if you want your program to do
different things depending on what happens?  That's where {\ttfamily IF...THEN}
comes in!

The {\ttfamily IF...THEN} statement lets your program make decisions.  It works
just like it sounds: {\em IF} something is true, {\em THEN} do something.

Let's try a simple example:

\codeblock{
	NEW\\\\
	10 CT = 0\\
	20 PRINT "COMMANDER X16"\\
	30 CT = CT + 1\\
	40 IF CT < 5 THEN GOTO 20\\
	50 PRINT "DONE!"\\
}

When you run this program, you'll see:

\begin{center}
\screenbox{3in}{2in}{
	COMMANDER X16\\
	COMMANDER X16\\
	COMMANDER X16\\
	COMMANDER X16\\
	COMMANDER X16\\
	DONE!\\\\
	READY.\\
	\cursor
}
\end{center}

Let's walk through what happened:

\begin{itemize}
	\item Line 10 sets a variable called {\ttfamily CT} (for ``count'') to 0
	\item Line 20 prints our message
	\item Line 30 adds 1 to {\ttfamily CT} (this counts how many times we've
		printed)
	\item Line 40 checks: is {\ttfamily CT} less than 5?  If yes, go back to
		line 20.  If no, continue to line 50
	\item Line 50 prints ``DONE!'' when the loop is finished
\end{itemize}

The magic is in line 40.  Each time through the loop, {\ttfamily CT} increases
by 1.  When {\ttfamily CT} finally reaches 5, the condition {\ttfamily CT < 5}
is no longer true, so the program continues to line 50 instead of jumping back.

\section{Comparison Symbols}

The {\ttfamily IF...THEN} statement can check many different conditions using
these comparison symbols:

\begin{center}
\begin{tabular}{|c|l|l|}
	\hline
	\textbf{Symbol} & \textbf{Meaning} & \textbf{Example} \\ \hline
	{\ttfamily <} & Less than & {\ttfamily IF X < 10} \\ \hline
	{\ttfamily >} & Greater than & {\ttfamily IF X > 10} \\ \hline
	{\ttfamily =} & Equal to & {\ttfamily IF X = 10} \\ \hline
	{\ttfamily <>} & Not equal to & {\ttfamily IF X <> 10} \\ \hline
	{\ttfamily <=} & Less than or equal to & {\ttfamily IF X <= 10} \\ \hline
	{\ttfamily >=} & Greater than or equal to & {\ttfamily IF X >= 10} \\ \hline
\end{tabular}
\end{center}

\section{IF...THEN with Strings}

You can also use {\ttfamily IF...THEN} to compare words and letters!  Try this
program:

\codeblock{
	NEW\\\\
	10 INPUT "WHAT IS THE PASSWORD"; P\$\\
	20 IF P\$ = "SECRET" THEN GOTO 50\\
	30 PRINT "WRONG! TRY AGAIN."\\
	40 GOTO 10\\
	50 PRINT "WELCOME, FRIEND!"\\
}

This program asks for a password.  If you type ``SECRET'', it welcomes you.
If you type anything else, it tells you to try again!

\tip{Shortcuts in IF...THEN}{
	You can leave out the word {\ttfamily GOTO} in an {\ttfamily IF...THEN}
	statement.  These two lines do exactly the same thing:\\

	{\ttfamily IF X = 5 THEN GOTO 100}\\
	{\ttfamily IF X = 5 THEN 100}\\

	Most programmers use the shorter version to save typing!
}

%----------------------------------------------------------------------------------------
%	CHAPTER - Loops with FOR...NEXT
%----------------------------------------------------------------------------------------

\chapter*{Loops with FOR...NEXT}\index{FOR...NEXT}\index{Loops}
\addcontentsline{toc}{chapter}{\protect\numberline{}Loops with FOR...NEXT}

Remember the program we wrote with {\ttfamily IF...THEN} to print ``COMMANDER
X16'' five times?  There's actually a simpler and more elegant way to do the
same thing using {\ttfamily FOR...NEXT}:

\codeblock{
	NEW\\\\
	10 FOR CT = 1 TO 5\\
	20 PRINT "COMMANDER X16"\\
	30 NEXT CT\\
}

Run this program and you'll see the same result---but with fewer lines of code!

Here's how {\ttfamily FOR...NEXT} works:

\begin{itemize}
	\item Line 10 says ``start {\ttfamily CT} at 1 and keep going until it
		reaches 5''
	\item Line 20 does the printing
	\item Line 30 says ``go to the {\ttfamily NEXT} value of {\ttfamily CT}''
		(which adds 1 to {\ttfamily CT} and jumps back to line 10)
\end{itemize}

The X16 automatically keeps track of {\ttfamily CT} for you, adding 1 each time
and stopping when it goes past 5.  Much simpler than doing it yourself!

\section{Counting by Different Amounts}

What if you want to count by something other than 1?  Use the {\ttfamily STEP}
keyword:

\codeblock{
	NEW\\\\
	10 FOR N = 2 TO 10 STEP 2\\
	20 PRINT N,\\
	30 NEXT N\\
}

This prints:

\begin{center}
\screenbox{3in}{0.75in}{
	\hspace{0.5em}2\hspace{2em}4\hspace{2em}6\hspace{2em}8\hspace{2em}10\\
}
\end{center}

You can even count backwards using a negative {\ttfamily STEP}:

\codeblock{
	NEW\\\\
	10 FOR N = 10 TO 1 STEP -1\\
	20 PRINT N\\
	30 NEXT N\\
	40 PRINT "BLAST OFF!"\\
}

This prints a countdown from 10 to 1, then ``BLAST OFF!''

\section{Using FOR...NEXT for Timing}

A common trick is to use a {\ttfamily FOR...NEXT} loop to create a delay in
your program:

\codeblock{
	10 PRINT "GET READY..."\\
	20 FOR D = 1 TO 1000 : NEXT D\\
	30 PRINT "GO!"\\
}

Line 20 counts from 1 to 1000, doing nothing useful---but it takes a moment to
do so!  This creates a pause between ``GET READY...'' and ``GO!''

\note{
	Notice that line 20 puts both {\ttfamily FOR} and {\ttfamily NEXT} on the
	same line, separated by a colon ({\ttfamily :}).  The colon lets you put
	multiple statements on one line.  This is handy for short loops like this
	delay.
}

%----------------------------------------------------------------------------------------
%	CHAPTER - Your First Game: Guess the Number
%----------------------------------------------------------------------------------------

\chapter*{Your First Game: Guess the Number}\index{Games}\index{Guessing Game}
\addcontentsline{toc}{chapter}{\protect\numberline{}Your First Game: Guess the Number}

Now let's put everything together and make a real game!  This is the program
from the beginning of the chapter.  Type it in and run it:

\codeblock{
	NEW\\\\
	10 REM GUESSING GAME\\
	20 N = INT(RND(1)*10)+1\\
	30 PRINT "GUESS A NUMBER FROM 1 TO 10:"\\
	40 INPUT G\\
	50 IF G = N THEN GOTO 80\\
	60 PRINT "TRY AGAIN!"\\
	70 GOTO 30\\
	80 PRINT "YOU GOT IT!"\\
}

When you run this game, the X16 picks a secret number between 1 and 10, and you
try to guess it.  Keep guessing until you get it right!

Let's trace through the program:

\begin{itemize}
	\item Line 10 is a {\ttfamily REM} (remark)---it's just a note to remind us
		what the program does.  The X16 ignores it.
	\item Line 20 picks a random number from 1 to 10 and stores it in
		{\ttfamily N}
	\item Line 30 asks for your guess
	\item Line 40 waits for you to type a number and stores it in {\ttfamily G}
	\item Line 50 checks if your guess equals the secret number.  If yes, jump
		to line 80 (you win!)
	\item Line 60 prints ``TRY AGAIN!'' (only reached if you didn't win)
	\item Line 70 goes back to ask for another guess
	\item Line 80 celebrates your victory!
\end{itemize}

\section{Making the Game Better}

This game works, but it could be more helpful.  What if the X16 told you
whether to guess higher or lower?  Try this improved version:

\codeblock{
	NEW\\\\
	10 REM IMPROVED GUESSING GAME\\
	20 N = INT(RND(1)*100)+1\\
	30 CT = 0\\
	40 PRINT "I'M THINKING OF A NUMBER"\\
	50 PRINT "FROM 1 TO 100."\\
	60 PRINT\\
	70 PRINT "WHAT IS YOUR GUESS";\\
	80 INPUT G\\
	90 CT = CT + 1\\
	100 IF G = N THEN GOTO 150\\
	110 IF G < N THEN PRINT "TOO LOW!"\\
	120 IF G > N THEN PRINT "TOO HIGH!"\\
	130 PRINT\\
	140 GOTO 70\\
	150 PRINT "YOU GOT IT IN"; CT; "GUESSES!"\\
}

This version:
\begin{itemize}
	\item Uses numbers from 1 to 100 (more challenging!)
	\item Tells you if your guess is too high or too low
	\item Counts how many guesses it takes you
\end{itemize}

\tip{The Best Strategy}{
	Here's a secret: the fastest way to guess the number is to always guess the
	middle of the remaining possibilities.  Start with 50.  If it's too high,
	guess 25.  If it's too low, guess 75.  Keep halving the range and you can
	always find a number from 1 to 100 in 7 guesses or less!
}

\section{Roll the Dice}

Here's another fun program that simulates rolling two dice:

\codeblock{
	NEW\\\\
	10 PRINT "PRESS ANY KEY TO ROLL..."\\
	20 GET A\$ : IF A\$ = "" THEN 20\\
	30 D1 = INT(6*RND(1))+1\\
	40 D2 = INT(6*RND(1))+1\\
	50 PRINT "DIE 1:"; D1\\
	60 PRINT "DIE 2:"; D2\\
	70 PRINT "TOTAL:"; D1+D2\\
	80 PRINT\\
	90 GOTO 10\\
}

This program waits for you to press a key, then ``rolls'' two dice and shows
you the results.  You can use this as part of a board game simulation!

Line 20 uses the {\ttfamily GET} statement, which checks if a key has been
pressed without waiting.  The {\ttfamily IF A\$ = "" THEN 20} part keeps
checking until you actually press something.

\section{What You've Learned}

Congratulations!  In this chapter, you've learned some of the most important
concepts in programming:

\begin{itemize}
	\item {\ttfamily RND} generates random numbers for games and simulations
	\item {\ttfamily INT} converts decimal numbers to whole numbers
	\item {\ttfamily IF...THEN} lets your programs make decisions
	\item {\ttfamily FOR...NEXT} repeats actions a specific number of times
	\item {\ttfamily REM} lets you add notes to your programs
	\item The colon ({\ttfamily :}) lets you put multiple statements on one
		line
\end{itemize}

With these tools, you can write all sorts of programs---games, quizzes,
simulations, and much more.  The only limit is your imagination!

\tryit{
	Try modifying the guessing game:
	\begin{itemize}
		\item Change the range to 1-1000 for a harder challenge
		\item Add a limit of 10 guesses (and print ``GAME OVER'' if they run
			out)
		\item Ask if the player wants to play again when the game ends
	\end{itemize}
}

\@openrighttrue\makeatother
