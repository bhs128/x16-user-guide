%----------------------------------------------------------------------------------------
%	PART - Sound and Music
%----------------------------------------------------------------------------------------

\makeatletter\@openrightfalse
\part{Sound and Music}

\xlargechaptertypein{
	\keybackgroundcolor{gray}
	\keytextcolor{black}
	10 FMINIT\\
	20 FMINST 0,0\\
	30 FMINST 1,0\\
	40 FMINST 2,0\\
	110 FMPLAY 0,"T140L6S1O5ED+ED+EO4BO5DC"\\
	120 FMCHORD1,"O4A"\\
	130 FMPLAY 0,"O1AO2EAO3CEA"\\
	140 FMCHORD1,"O3B"\\
	150 FMPLAY 0,"O1EO2EG+O3EG+B"\\
	160 FMCHORD1,"O4C"\\
	170 FMPLAY 0,"O1AO2EAO3EO5ED+"\\
	180 FMPLAY 0,"O5ED+EO4BO5DC"\\
	190 FMCHORD1,"O3A"\\
	200 FMPLAY 0,"O1AO2EAO3CEA"\\
	210 FMCHORD1,"O3B"\\
	220 FMPLAY 0,"L6O1EO2EG+O3EO5CO4B"\\
	230 FMCHORD 0,"L1 O1AO3C+A"\\
}

%----------------------------------------------------------------------------------------
%	CHAPTER - Sound and Music
%----------------------------------------------------------------------------------------

\chapter*{Sound and Music}\index{Sound}\index{Music}
\addcontentsline{toc}{chapter}{\protect\numberline{}Sound and Music}

Did you type in the program at the beginning of this chapter?  If so, you heard
the Commander X16 play the opening of Beethoven's ``Für Elise''---complete with
melody and accompaniment!  Even if you don't know how to read music, you can
make the X16 create musical masterpieces by following a few simple rules.\\

The Commander X16 has incredible sound capabilities that rival professional
music synthesizers.  With not one, but \textit{two} sound chips, you can create
everything from simple beeps and boops to complex multi-voice orchestrations.
The first chip is called the \textbf{PSG} (Programmable Sound Generator), which
is built into the VERA graphics chip and provides 16 voices of simple waveform
sounds.  The second is the \textbf{YM2151}, a dedicated FM synthesis chip with
8 channels that can produce rich, instrument-like sounds.\\

Don't worry if those terms sound complicated---the X16's BASIC makes it easy to
create music without understanding the technical details!

%----------------------------------------------------------------------------------------
%	CHAPTER - Making Your First Sound
%----------------------------------------------------------------------------------------

\chapter*{Making Your First Sound}\index{Sound!first sound}
\addcontentsline{toc}{chapter}{\protect\numberline{}Making Your First Sound}

Let's start by making a simple sound using the PSG chip.  Clear any old program
by holding \shiftkey and pressing \clrhomekey, then type {\ttfamily NEW} and
press \returnkey.  Now type in this short program:\\

\codeblock{
	10 PSGINIT\\
	20 PSGNOTE 0,\$4A\\
	30 SLEEP 120\\
	40 PSGNOTE 0,0\\
}

Type {\ttfamily RUN} and press \returnkey.  You should hear a tone play for
about two seconds and then stop.  Congratulations---you just made your first
sound on the Commander X16!\\

Let's look at what each line does:

\begin{itemize}
	\item \textbf{Line 10}: {\ttfamily PSGINIT} initializes (prepares) the PSG
		sound chip.  You should always run this command before making sounds
		with the PSG.
	\item \textbf{Line 20}: {\ttfamily PSGNOTE 0,\$4A} tells voice 0 to play a
		note.  The {\ttfamily \$4A} is a special code that means ``Concert A''
		(the note that orchestras tune to).
	\item \textbf{Line 30}: {\ttfamily SLEEP 120} pauses for about 2 seconds
		(60 ``jiffies'' = 1 second).  This gives the note time to play.
	\item \textbf{Line 40}: {\ttfamily PSGNOTE 0,0} turns off the note on voice 0.
\end{itemize}

\tip{Hexadecimal Numbers}{The {\ttfamily \$} symbol means the number is written
in \textit{hexadecimal} (base 16).  Don't worry about understanding hex right
now---just type the codes exactly as shown!}

Now let's try the FM chip.  Type {\ttfamily NEW} and enter this program:\\

\codeblock{
	10 FMINIT\\
	20 FMNOTE 0,\$4A\\
	30 SLEEP 120\\
	40 FMNOTE 0,0\\
}

Type {\ttfamily RUN} and listen.  Notice how different it sounds from the PSG?
The FM chip creates richer, more complex tones that can sound like real musical
instruments.  The {\ttfamily FMINIT} command initializes the FM chip and loads
a default instrument (an electric piano sound) into all 8 channels.\\

Did you notice another difference?  The PSG note played at a constant volume
until we turned it off, but the FM note \textit{faded away on its own}!  This
is because FM instruments have built-in ``envelope'' settings that control how
the sound changes over time---just like a real piano note that gradually fades
after you press the key.\\

To make an FM note sustain longer, you can use an instrument with a longer
envelope.  Try this program with an organ sound that holds its notes:\\

\codeblock{
	10 FMINIT\\
	20 FMINST 0,16:REM ORGAN\\
	30 FMNOTE 0,\$4A\\
	40 SLEEP 180\\
	50 FMNOTE 0,0\\
}

The organ sustains until you release it, just like the PSG.  Different
instruments have different behaviors---experiment to find the sounds you like!\\

%----------------------------------------------------------------------------------------
%	CHAPTER - The Voices of the X16
%----------------------------------------------------------------------------------------

\chapter*{The Voices of the X16}\index{Sound!voices}
\addcontentsline{toc}{chapter}{\protect\numberline{}The Voices of the X16}

Just like a choir has many singers, the Commander X16 has many ``voices'' that
can play sounds at the same time.

\section{PSG Voices}

The PSG chip has \textbf{16 voices}, numbered 0 through 15.  Each voice can
play one note at a time, and you can use multiple voices together to create
chords or complex sound effects.\\

Each PSG voice can produce four different types of sound waves:

\begin{itemize}
	\item \textbf{Pulse (Square) Wave}: A buzzy, electronic sound---great for
		classic video game music
	\item \textbf{Sawtooth Wave}: A bright, brassy sound
	\item \textbf{Triangle Wave}: A soft, mellow sound like a flute
	\item \textbf{Noise}: Random sounds for explosions, drums, or wind effects
\end{itemize}

The default waveform after {\ttfamily PSGINIT} is a pulse (square) wave with a
50\% duty cycle.  You can change the waveform using the {\ttfamily PSGWAV}
command.  Try this program to hear the different waveforms:\\

\codeblock{
	10 PSGINIT\\
	20 PRINT "PULSE WAVE"\\
	30 PSGWAV 0,63:REM PULSE 50\%\\
	40 PSGNOTE 0,\$4A\\
	50 SLEEP 90:PSGNOTE 0,0\\
	60 SLEEP 30\\
	70 PRINT "SAWTOOTH WAVE"\\
	80 PSGWAV 0,64:REM SAWTOOTH\\
	90 PSGNOTE 0,\$4A\\
	100 SLEEP 90:PSGNOTE 0,0\\
	110 SLEEP 30\\
	120 PRINT "TRIANGLE WAVE"\\
	130 PSGWAV 0,128:REM TRIANGLE\\
	140 PSGNOTE 0,\$4A\\
	150 SLEEP 90:PSGNOTE 0,0\\
	160 SLEEP 30\\
	170 PRINT "NOISE"\\
	180 PSGWAV 0,192:REM NOISE\\
	190 PSGNOTE 0,\$4A\\
	200 SLEEP 90:PSGNOTE 0,0\\
}

\section{FM Channels}

The YM2151 FM chip has \textbf{8 channels}, numbered 0 through 7.  Unlike the
simple waveforms of the PSG, the FM chip uses a technique called \textit{frequency
modulation} to create complex, rich sounds that can imitate real instruments.\\

The X16 comes with over 160 built-in instrument sounds (called ``patches'')
that you can load into any FM channel using the {\ttfamily FMINST} command.
Here are just a few examples:\\

\begin{center}
\begin{tabular}{|c|l||c|l|}
	\hline
	\textbf{Number} & \textbf{Instrument} & \textbf{Number} & \textbf{Instrument}\\ \hline
	0 & Electric Piano 1 & 24 & Acoustic Guitar (nylon)\\ \hline
	16 & Drawbar Organ & 40 & Violin\\ \hline
	56 & Trumpet & 73 & Flute\\ \hline
	80 & Square Lead & 104 & Sitar\\ \hline
	127 & Applause & 11 & Vibraphone\\ \hline
\end{tabular}
\end{center}

\vspace{16pt}

Try this program to hear some different instruments:\\

\codeblock{
	10 FMINIT\\
	20 PRINT "ELECTRIC PIANO"\\
	30 FMINST 0,0\\
	40 FMNOTE 0,\$4A:SLEEP 90\\
	50 FMNOTE 0,0:SLEEP 30\\
	60 PRINT "TRUMPET"\\
	70 FMINST 0,56\\
	80 FMNOTE 0,\$4A:SLEEP 90\\
	90 FMNOTE 0,0:SLEEP 30\\
	100 PRINT "FLUTE"\\
	110 FMINST 0,73\\
	120 FMNOTE 0,\$4A:SLEEP 90\\
	130 FMNOTE 0,0:SLEEP 30\\
	140 PRINT "VIBRAPHONE"\\
	150 FMINST 0,11\\
	160 FMNOTE 0,\$4A:SLEEP 90\\
	170 FMNOTE 0,0\\
}

\note{A complete list of FM instrument patches is available in the Appendix.
Try experimenting with different patch numbers to find sounds you like!}

%----------------------------------------------------------------------------------------
%	CHAPTER - Making Music with PLAY Commands
%----------------------------------------------------------------------------------------

\chapter*{Making Music with PLAY Commands}\index{Sound!FMPLAY}\index{Sound!PSGPLAY}
\addcontentsline{toc}{chapter}{\protect\numberline{}Making Music with PLAY Commands}

While {\ttfamily PSGNOTE} and {\ttfamily FMNOTE} let you play individual notes,
the Commander X16 has a much easier way to play melodies: the {\ttfamily FMPLAY}
and {\ttfamily PSGPLAY} commands.\\

These commands let you write music using letter names for notes, just like
reading sheet music!  Type in this program:\\

\codeblock{
	10 FMINIT\\
	20 FMPLAY 0,"CDEFGAB>C"\\
}

Run it and you'll hear a C major scale!  The letters C, D, E, F, G, A, and B
play those musical notes.  The {\ttfamily >} symbol moves up one octave, so
the final C is higher than where we started.\\

\section{Understanding the Music String}

The string you pass to {\ttfamily FMPLAY} or {\ttfamily PSGPLAY} is like a
simple music notation.  Here are the basic codes:\\

\begin{center}
\begin{tabular}{|c|p{10cm}|}
	\hline
	\textbf{Code} & \textbf{What It Does}\\ \hline
	A-G & Plays that musical note\\ \hline
	+ or \# & Sharp (raises the note by a half step)---put after the note letter\\ \hline
	- & Flat (lowers the note by a half step)---put after the note letter\\ \hline
	R & Rest (silence)\\ \hline
	> & Move up one octave\\ \hline
	< & Move down one octave\\ \hline
	O{\em n} & Set the octave to {\em n} (0-7), where 4 is the middle octave\\ \hline
	L{\em n} & Set the default note length (1=whole, 2=half, 4=quarter, 8=eighth, etc.)\\ \hline
	T{\em n} & Set the tempo (beats per minute)\\ \hline
	. & Dotted note (adds 50\% more time to the note)\\ \hline
\end{tabular}
\end{center}

\vspace{16pt}

Let's try a more interesting example.  This program plays ``Mary Had a Little
Lamb'':\\

\codeblock{
	10 FMINIT\\
	20 FMPLAY 0,"T120 L4 O4"\\
	30 FMPLAY 0,"EDCDEEE2"\\
	40 FMPLAY 0,"DDD2EGG2"\\
	50 FMPLAY 0,"EDCDEEEE"\\
	60 FMPLAY 0,"DDEDL2C"\\
}

Let's break down what's happening:

\begin{itemize}
	\item {\ttfamily T120} sets the tempo to 120 beats per minute
	\item {\ttfamily L4} sets the default note length to quarter notes
	\item {\ttfamily O4} sets the octave to 4 (middle octave)
	\item Numbers after notes (like {\ttfamily E2}) set that note's length:
		{\ttfamily 2} means a half note, which is twice as long as a quarter note
\end{itemize}

\section{Playing Chords}

What if you want to play multiple notes at the same time?  You can use the
{\ttfamily FMCHORD} or {\ttfamily PSGCHORD} commands!  Try this:\\

\codeblock{
	10 FMINIT\\
	20 FMINST 0,16:FMINST 1,16:FMINST 2,16\\
	30 FMCHORD 0,"O3CEG"\\
	40 SLEEP 180\\
	50 FMCHORD 0,"RRR"\\
}

This plays a C major chord by starting notes C, E, and G simultaneously on
channels 0, 1, and 2.  The {\ttfamily RRR} at the end releases all three notes.\\

\tip{Mixing Melody and Chords}{You can mix melody and chords!  Use
{\ttfamily FMCHORD} to start background chords on some channels, then use
{\ttfamily FMPLAY} on another channel to play a melody over the top---just like
in the opening ``Für Elise'' example!}

%----------------------------------------------------------------------------------------
%	CHAPTER - Sound Effects
%----------------------------------------------------------------------------------------

\chapter*{Sound Effects}\index{Sound!effects}
\addcontentsline{toc}{chapter}{\protect\numberline{}Sound Effects}

Music isn't the only thing you can create with sound.  Games and programs often
need sound effects like explosions, laser beams, or warning buzzers.  The
Commander X16 makes it easy to create these too!\\

\section{A Simple Beep}

Here's a basic beep that could be used for a button press or menu selection:\\

\codeblock{
	10 PSGINIT\\
	20 PSGNOTE 0,\$5C:REM HIGH C\\
	30 SLEEP 10\\
	40 PSGNOTE 0,0\\
}

\section{An Explosion}

Explosions sound great with the noise waveform.  Try this:\\

\codeblock{
	10 PSGINIT\\
	20 PSGWAV 0,192:REM NOISE\\
	30 PSGVOL 0,63:REM FULL VOLUME\\
	40 PSGNOTE 0,\$30\\
	50 FOR V=63 TO 0 STEP -1\\
	60 PSGVOL 0,V\\
	70 FOR T=1 TO 20:NEXT T\\
	80 NEXT V\\
}

This creates an explosion sound by starting with loud noise and gradually
reducing the volume to create a fade-out effect.  The {\ttfamily PSGVOL}
command sets the volume of a voice (0 is silent, 63 is maximum).\\

\section{A Laser Sound}

Science fiction games need laser sounds!  This effect uses a rapidly falling
pitch:\\

\codeblock{
	10 PSGINIT\\
	20 FOR F=4000 TO 200 STEP -100\\
	30 PSGFREQ 0,F\\
	40 NEXT F\\
	50 PSGFREQ 0,0\\
}

The {\ttfamily PSGFREQ} command lets you specify the exact frequency in Hertz
(cycles per second) instead of using note codes.  By quickly decreasing the
frequency from high to low, we create that classic ``pew pew'' laser sound.\\

\section{A Siren}

Emergency!  Here's a two-tone siren:\\

\codeblock{
	10 PSGINIT\\
	20 FOR I=1 TO 5\\
	30 PSGFREQ 0,800\\
	40 SLEEP 25\\
	50 PSGFREQ 0,600\\
	60 SLEEP 25\\
	70 NEXT I\\
	80 PSGFREQ 0,0\\
}

\section{Power-Up Sound}

When the player collects a power-up in a game, play this rising arpeggio:\\

\codeblock{
	10 PSGINIT\\
	20 PSGWAV 0,63:REM PULSE\\
	30 PSGPLAY 0,"T250 L32 O4 CEG>CEG>C"\\
}

Notice how we used {\ttfamily PSGPLAY} here---you can use the play commands for
sound effects too, not just music!

\tryit{Try combining different waveforms, volumes, and frequencies to create
your own unique sound effects.  What sound would a jumping character make?
A door opening?  A coin being collected?}

%----------------------------------------------------------------------------------------
%	CHAPTER - Using the X16 as a Piano
%----------------------------------------------------------------------------------------

\chapter*{Using the X16 as a Piano}\index{Sound!piano}
\addcontentsline{toc}{chapter}{\protect\numberline{}Using the X16 as a Piano}

Now let's put everything together and turn your Commander X16 into a musical
instrument!  This program lets you play notes using the keyboard:\\

\codeblock{
	10 FMINIT\\
	20 PRINT "\shiftkey\clrhomekey"\\
	30 PRINT "*** X16 PIANO ***"\\
	40 PRINT\\
	50 PRINT "PLAY NOTES WITH KEYS:"\\
	60 PRINT " A S D F G H J K"\\
	70 PRINT " C D E F G A B C"\\
	80 PRINT\\
	90 PRINT "PRESS Q TO QUIT"\\
	100 PRINT\\
	110 GET K\$:IF K\$="" THEN 110\\
	120 IF K\$="Q" THEN FMINIT:END\\
	130 IF K\$="A" THEN N\$="C"\\
	140 IF K\$="S" THEN N\$="D"\\
	150 IF K\$="D" THEN N\$="E"\\
	160 IF K\$="F" THEN N\$="F"\\
	170 IF K\$="G" THEN N\$="G"\\
	180 IF K\$="H" THEN N\$="A"\\
	190 IF K\$="J" THEN N\$="B"\\
	200 IF K\$="K" THEN N\$=">C"\\
	210 IF N\$="" THEN 110\\
	220 PRINT N\$;" ";\\
	230 FMPLAY 0,"L8 O4 "+N\$\\
	240 N\$=""\\
	250 GOTO 110\\
}

When you run this program, you can play notes by pressing the keys A through K
on your keyboard.  Each key corresponds to a note of the C major scale.  Press
Q when you're done playing.\\

\section{Adding More Features}

Here's an enhanced version with sharps, flats, and octave control:\\

\codeblock{
	10 FMINIT:FMINST 0,0\\
	20 O=4:REM STARTING OCTAVE\\
	30 PRINT "\shiftkey\clrhomekey"\\
	40 PRINT "*** DELUXE X16 PIANO ***"\\
	50 PRINT\\
	60 PRINT "WHITE KEYS: A S D F G H J K"\\
	70 PRINT "            C D E F G A B C"\\
	80 PRINT\\
	90 PRINT "BLACK KEYS: W E   T Y U"\\
	100 PRINT "            C\# D\#  F\# G\# A\#"\\
	110 PRINT\\
	120 PRINT "1-7 = OCTAVE, Q = QUIT"\\
	130 PRINT\\
	140 GET K\$:IF K\$="" THEN 140\\
	150 IF K\$="Q" THEN FMINIT:END\\
	160 IF K\$>="1" AND K\$<="7" THEN O=VAL(K\$):PRINT "OCTAVE:";O:GOTO 140\\
	170 N\$=""\\
	180 IF K\$="A" THEN N\$="C"\\
	190 IF K\$="W" THEN N\$="C+"\\
	200 IF K\$="S" THEN N\$="D"\\
	210 IF K\$="E" THEN N\$="D+"\\
	220 IF K\$="D" THEN N\$="E"\\
	230 IF K\$="F" THEN N\$="F"\\
	240 IF K\$="T" THEN N\$="F+"\\
	250 IF K\$="G" THEN N\$="G"\\
	260 IF K\$="Y" THEN N\$="G+"\\
	270 IF K\$="H" THEN N\$="A"\\
	280 IF K\$="U" THEN N\$="A+"\\
	290 IF K\$="J" THEN N\$="B"\\
	300 IF K\$="K" THEN N\$=">C"\\
	310 IF N\$="" THEN 140\\
	320 PRINT N\$;" ";\\
	330 FMPLAY 0,"L8 O"+STR\$(O)+N\$\\
	340 GOTO 140\\
}

Now you can play sharps using the ``black key'' row (W, E, T, Y, U) and change
octaves by pressing number keys 1-7!\\

\tryit{Try playing some simple songs on your X16 piano.  Can you figure out
``Twinkle Twinkle Little Star'' or ``Happy Birthday''?  Once you learn a song,
you can write it down as an {\ttfamily FMPLAY} string!}

%----------------------------------------------------------------------------------------
%	CHAPTER - Playing Songs
%----------------------------------------------------------------------------------------

\chapter*{Playing Songs}\index{Sound!songs}
\addcontentsline{toc}{chapter}{\protect\numberline{}Playing Songs}

Let's learn how to play complete songs on the Commander X16.  We'll start with
some simple melodies and work up to multi-voice arrangements.\\

\section{Happy Birthday}

Here's the classic birthday song:\\

\codeblock{
	10 FMINIT\\
	20 FMINST 0,11:REM VIBRAPHONE\\
	30 FMVIB 200,20:REM ADD VIBRATO\\
	40 FMPLAY 0,"T100 L8 O4"\\
	50 FMPLAY 0,"C.C16DL4C"\\
	60 FMPLAY 0,"L4 F E2"\\
	70 FMPLAY 0,"L8 C.C16DL4C"\\
	80 FMPLAY 0,"L4 G F2"\\
	90 FMPLAY 0,"L8 C.C16>L4C"\\
	100 FMPLAY 0,"L4 <A F E D"\\
	110 FMPLAY 0,"L8 A-.A-16L4GFG"\\
	120 FMPLAY 0,"L1 F"\\
}

The {\ttfamily FMVIB} command adds a gentle vibrato effect that makes the
vibraphone sound more realistic.\\

\section{Twinkle Twinkle Little Star}

This nursery rhyme is great for beginners:\\

\codeblock{
	10 FMINIT\\
	20 FMINST 0,13:REM XYLOPHONE\\
	30 FMPLAY 0,"T120 L4 O4"\\
	40 FMPLAY 0,"CCGGAAG2"\\
	50 FMPLAY 0,"FFEEDDC2"\\
	60 FMPLAY 0,"GGFFEED2"\\
	70 FMPLAY 0,"GGFFEED2"\\
	80 FMPLAY 0,"CCGGAAG2"\\
	90 FMPLAY 0,"FFEEDDC2"\\
}

\section{A Two-Voice Arrangement}

Here's ``Frère Jacques'' (Are You Sleeping?) with a simple two-voice
arrangement---melody and bass:\\

\codeblock{
	10 FMINIT\\
	20 FMINST 0,0:REM PIANO FOR MELODY\\
	30 FMINST 1,32:REM BASS\\
	40 FMVOL 1,45:REM BASS QUIETER\\
	50 REM === PHRASE 1 ===\\
	60 FMCHORD 1,"O2C"\\
	70 FMPLAY 0,"T140 L4 O4 CDEC"\\
	80 FMCHORD 1,"R"\\
	90 REM === PHRASE 2 ===\\
	100 FMCHORD 1,"O2C"\\
	110 FMPLAY 0,"CDEC"\\
	120 FMCHORD 1,"R"\\
	130 REM === PHRASE 3 ===\\
	140 FMCHORD 1,"O2G"\\
	150 FMPLAY 0,"EFG2"\\
	160 FMCHORD 1,"O2C"\\
	170 FMPLAY 0,"EFG2"\\
	180 FMCHORD 1,"R"\\
	190 REM === PHRASE 4 ===\\
	200 FMCHORD 1,"O2G"\\
	210 FMPLAY 0,"L8G AG FEL4C"\\
	220 FMCHORD 1,"O2C"\\
	230 FMPLAY 0,"L8G AG FEL4C"\\
	240 FMCHORD 1,"R"\\
	250 REM === PHRASE 5 ===\\
	260 FMCHORD 1,"O2F"\\
	270 FMPLAY 0,"L4 <G>C2"\\
	280 FMCHORD 1,"O2C"\\
	290 FMPLAY 0,"L4 <G>C2"\\
	300 FMCHORD 1,"R"\\
}

This example shows how to combine {\ttfamily FMCHORD} and {\ttfamily FMPLAY} to
create arrangements where one instrument holds a note (the bass) while another
plays a melody (the piano).\\

\section{Using DATA for Songs}

For longer songs, you can store the notes in {\ttfamily DATA} statements:\\

\codeblock{
	10 FMINIT\\
	20 FMINST 0,73:REM FLUTE\\
	30 FMPLAY 0,"T100 L4 O5"\\
	40 READ N\$\\
	50 IF N\$="END" THEN END\\
	60 FMPLAY 0,N\$\\
	70 GOTO 40\\
	100 DATA "E2DC","L2<B>E","L4DC<B"\\
	110 DATA "L2AB","L4>CDC","L2<BA"\\
	120 DATA "END"\\
}

This technique makes it easy to compose longer pieces---just add more
{\ttfamily DATA} lines!\\

\tip{Saving Your Music}{Once you've created a song you like, you can save your
program to the SD card using {\ttfamily SAVE "MYSONG.BAS"}.  Then you can load
and play it anytime with {\ttfamily LOAD "MYSONG.BAS"} followed by
{\ttfamily RUN}.}

%----------------------------------------------------------------------------------------
%	CHAPTER - A Few Words About Sound Commands
%----------------------------------------------------------------------------------------

\chapter*{A Few Words About Sound Commands}\index{Sound!commands summary}
\addcontentsline{toc}{chapter}{\protect\numberline{}A Few Words About Sound Commands}

Here's a quick reference of all the sound commands we've learned:\\

\section{PSG Commands}

\begin{tabular}{|l|p{9cm}|}
	\hline
	\textbf{Command} & \textbf{What It Does}\\ \hline
	{\ttfamily PSGINIT} & Initialize the PSG chip (always do this first!)\\ \hline
	{\ttfamily PSGNOTE v,n} & Play note {\em n} on voice {\em v} (0 = stop)\\ \hline
	{\ttfamily PSGFREQ v,f} & Play frequency {\em f} Hz on voice {\em v}\\ \hline
	{\ttfamily PSGVOL v,vol} & Set volume of voice {\em v} (0-63)\\ \hline
	{\ttfamily PSGWAV v,w} & Set waveform of voice {\em v}\\ \hline
	{\ttfamily PSGPAN v,p} & Set stereo panning (1=left, 2=right, 3=both)\\ \hline
	{\ttfamily PSGPLAY v,s\$} & Play music string {\em s\$} on voice {\em v}\\ \hline
	{\ttfamily PSGCHORD v,s\$} & Start chord from string {\em s\$} on voices starting at {\em v}\\ \hline
\end{tabular}

\section{FM Commands}

\begin{tabular}{|l|p{9cm}|}
	\hline
	\textbf{Command} & \textbf{What It Does}\\ \hline
	{\ttfamily FMINIT} & Initialize the FM chip and load default instruments\\ \hline
	{\ttfamily FMINST c,p} & Load instrument patch {\em p} into channel {\em c}\\ \hline
	{\ttfamily FMNOTE c,n} & Play note {\em n} on channel {\em c} (0 = stop)\\ \hline
	{\ttfamily FMFREQ c,f} & Play frequency {\em f} Hz on channel {\em c}\\ \hline
	{\ttfamily FMVOL c,vol} & Set volume of channel {\em c} (0-63)\\ \hline
	{\ttfamily FMPAN c,p} & Set stereo panning (1=left, 2=right, 3=both)\\ \hline
	{\ttfamily FMVIB s,d} & Set vibrato speed {\em s} and depth {\em d}\\ \hline
	{\ttfamily FMPLAY c,s\$} & Play music string {\em s\$} on channel {\em c}\\ \hline
	{\ttfamily FMCHORD c,s\$} & Start chord from string {\em s\$} on channels starting at {\em c}\\ \hline
	{\ttfamily FMDRUM c,d} & Play drum sound {\em d} on channel {\em c}\\ \hline
\end{tabular}

\vspace{16pt}

For a complete reference of music string macros, see the Appendix on ``Macro
Language for Music.''\\

Now you know enough about sound and music to add audio to your programs, create
your own songs, and even build musical instruments!  The Commander X16's dual
sound chips give you incredible creative possibilities---so experiment and have
fun making noise!

\@openrighttrue\makeatother
