%----------------------------------------------------------------------------------------
%	APPENDIX - Error Messages
%----------------------------------------------------------------------------------------

\chapter*{Error Messages}\index{Error Messages}
\addcontentsline{toc}{chapter}{\protect\numberline{}Error Messages}

\section*{Introduction}

When the Commander X16 encounters a problem while running your program, it
displays an error message. Understanding these messages will help you fix
problems in your programs quickly.\\

Error messages in BASIC are displayed in the format:

\codeblock{
	?ERROR NAME  IN LINE xxx\\
}

The ``xxx'' is the line number where the error occurred. If you are typing
commands in direct mode (without line numbers), the ``IN LINE xxx'' part
will not appear.\\

Here is a complete list of the error messages you may encounter on the
Commander X16, along with explanations of what causes them and how to fix them.

%----------------------------------------------------------------------------------------
\section*{BASIC Error Messages}
%----------------------------------------------------------------------------------------

\subsection*{?BAD DATA ERROR}
\index{Error Messages!BAD DATA}

String data was received from an open file, but the program was expecting
numeric data. Check that your file contains the type of data your program
expects, and that you are reading it in the correct order.

\subsection*{?BAD SUBSCRIPT ERROR}
\index{Error Messages!BAD SUBSCRIPT}

The program tried to access an element of an array using a subscript (index
number) that is outside the range specified in the {\ttfamily DIM} statement.
For example, if you {\ttfamily DIM A(10)} and then try to access {\ttfamily A(15)},
you will get this error. Array subscripts must be between 0 and the dimension
you specified.

\subsection*{?BREAK}
\index{Error Messages!BREAK}

Program execution was stopped because you pressed the \runstopkey key.
This is not really an error---it's a way to interrupt a running program.
You can use {\ttfamily CONT} to continue running the program from where
it stopped.

\subsection*{?CAN'T CONTINUE ERROR}
\index{Error Messages!CAN'T CONTINUE}

The {\ttfamily CONT} command cannot continue the program. This happens when:
\begin{itemize}
	\item The program was never {\ttfamily RUN}
	\item The program stopped due to an error
	\item You edited any line of the program (even pressing \returnkey on an unchanged line counts as editing)
	\item You caused an error before typing {\ttfamily CONT}
\end{itemize}
To restart the program, use {\ttfamily RUN} instead.

\subsection*{?DEVICE NOT PRESENT ERROR}
\index{Error Messages!DEVICE NOT PRESENT}

The device number you specified in an {\ttfamily OPEN}, {\ttfamily CLOSE},
{\ttfamily CMD}, {\ttfamily PRINT\#}, {\ttfamily INPUT\#}, or {\ttfamily GET\#}
statement is not available. Make sure the device is connected and turned on,
and that you are using the correct device number (typically 8 for the SD card).

\subsection*{?DIVISION BY ZERO ERROR}
\index{Error Messages!DIVISION BY ZERO}

You attempted to divide a number by zero. Division by zero is mathematically
undefined and not allowed. Check your calculations to make sure the divisor
is never zero. For example:

\codeblock{
	10 X = 0\\
	20 PRINT 10/X\\
}

This will cause a {\ttfamily ?DIVISION BY ZERO ERROR} because {\ttfamily X} is zero.

\subsection*{?EXTRA IGNORED}
\index{Error Messages!EXTRA IGNORED}

Too many items of data were typed in response to an {\ttfamily INPUT} statement.
Only the first few items were accepted, and the rest were ignored. This is a
warning rather than a fatal error---your program will continue running.

\subsection*{?FILE NOT FOUND ERROR}
\index{Error Messages!FILE NOT FOUND}

The file you tried to {\ttfamily LOAD} or {\ttfamily OPEN} for reading does
not exist on the SD card. Check the spelling of the filename and make sure
the file is in the correct directory. You can use the {\ttfamily DOS"\$"}
command to list the files on the SD card.

\subsection*{?FILE NOT OPEN ERROR}
\index{Error Messages!FILE NOT OPEN}

The file number specified in a {\ttfamily CLOSE}, {\ttfamily CMD},
{\ttfamily PRINT\#}, {\ttfamily INPUT\#}, or {\ttfamily GET\#} statement
was not previously opened with an {\ttfamily OPEN} statement. Make sure you
{\ttfamily OPEN} a file before trying to use it.

\subsection*{?FILE OPEN ERROR}
\index{Error Messages!FILE OPEN}

You attempted to {\ttfamily OPEN} a file using a file number that is already
in use. Each open file must have a unique file number. Either {\ttfamily CLOSE}
the existing file first, or use a different file number.

\subsection*{?FORMULA TOO COMPLEX ERROR}
\index{Error Messages!FORMULA TOO COMPLEX}

The expression you are trying to evaluate is too complicated for the X16 to
handle all at once. Try breaking it into smaller parts by storing intermediate
results in variables. This can also occur if you have too many nested
parentheses.

\subsection*{?ILLEGAL DIRECT ERROR}
\index{Error Messages!ILLEGAL DIRECT}

You tried to use a statement in direct mode (without a line number) that can
only be used within a program. The {\ttfamily INPUT} and {\ttfamily DEF FN}
statements cannot be used in direct mode---they must be part of a program.

\subsection*{?ILLEGAL QUANTITY ERROR}
\index{Error Messages!ILLEGAL QUANTITY}

A number used as an argument to a function or statement is out of the
allowable range. For example:
\begin{itemize}
	\item {\ttfamily CHR\$(300)} --- the argument must be 0--255
	\item {\ttfamily POKE 65536,0} --- the address must be 0--65535
	\item {\ttfamily LOG(0)} --- logarithm requires a positive number
	\item {\ttfamily SQR(-1)} --- square root requires a non-negative number
\end{itemize}

\subsection*{?LOAD ERROR}
\index{Error Messages!LOAD}

There was a problem loading the program. The file may be corrupted or the
wrong type. Try loading the file again, or check that the file is a valid
BASIC program.

\subsection*{?NEXT WITHOUT FOR ERROR}
\index{Error Messages!NEXT WITHOUT FOR}

A {\ttfamily NEXT} statement was encountered without a matching {\ttfamily FOR}
statement. This is usually caused by:
\begin{itemize}
	\item Incorrectly nested loops
	\item A variable name in {\ttfamily NEXT} that doesn't match any {\ttfamily FOR} loop
	\item Jumping into the middle of a {\ttfamily FOR...NEXT} loop with {\ttfamily GOTO}
\end{itemize}

\subsection*{?NOT INPUT FILE ERROR}
\index{Error Messages!NOT INPUT FILE}

You tried to {\ttfamily INPUT} or {\ttfamily GET} data from a file that was
opened for output only. When opening a file, you must specify the correct
mode for how you intend to use it.

\subsection*{?NOT OUTPUT FILE ERROR}
\index{Error Messages!NOT OUTPUT FILE}

You tried to {\ttfamily PRINT} data to a file that was opened for input only.
When opening a file, you must specify the correct mode for how you intend to
use it.

\subsection*{?OUT OF DATA ERROR}
\index{Error Messages!OUT OF DATA}

A {\ttfamily READ} statement was executed, but there is no more data left
in the {\ttfamily DATA} statements. Make sure you have enough data items
for all your {\ttfamily READ} statements, or use {\ttfamily RESTORE} to
start reading from the beginning again.

\subsection*{?OUT OF MEMORY ERROR}
\index{Error Messages!OUT OF MEMORY}

There is no more RAM available for your program or variables. This can occur
when:
\begin{itemize}
	\item Your program is too large
	\item You have too many variables or strings
	\item Too many {\ttfamily FOR} loops are nested
	\item Too many {\ttfamily GOSUB}s are active (without {\ttfamily RETURN})
\end{itemize}
Try shortening your program, using fewer variables, or simplifying your logic.

\subsection*{?OVERFLOW ERROR}
\index{Error Messages!OVERFLOW}

The result of a calculation is larger than the largest number the X16 can
handle, which is approximately 1.70141884 $\times$ 10\textsuperscript{38}.
Try using smaller numbers or breaking your calculation into steps.

\subsection*{?REDIM'D ARRAY ERROR}
\index{Error Messages!REDIM'D ARRAY}

An array can only be dimensioned once with {\ttfamily DIM}. If you use an
array variable before dimensioning it, BASIC automatically creates it with
10 elements. Any later {\ttfamily DIM} for the same array causes this error.
Put all your {\ttfamily DIM} statements at the beginning of your program
to avoid this problem.

\subsection*{?REDO FROM START}
\index{Error Messages!REDO FROM START}

You typed text when the {\ttfamily INPUT} statement was expecting a number.
This is not a fatal error---simply retype your response with a valid number,
and the program will continue.

\subsection*{?RETURN WITHOUT GOSUB ERROR}
\index{Error Messages!RETURN WITHOUT GOSUB}

A {\ttfamily RETURN} statement was encountered, but no {\ttfamily GOSUB}
had been executed to call a subroutine. Make sure every {\ttfamily RETURN}
has a matching {\ttfamily GOSUB}, and avoid jumping into subroutines with
{\ttfamily GOTO}.

\subsection*{?STRING TOO LONG ERROR}
\index{Error Messages!STRING TOO LONG}

A string can contain at most 255 characters. Your string operation created
a result longer than 255 characters. Try working with shorter strings or
breaking your text into multiple string variables.

\subsection*{?SYNTAX ERROR}
\index{Error Messages!SYNTAX ERROR}

The Commander X16 doesn't understand what you typed. Common causes include:
\begin{itemize}
	\item Misspelled BASIC keywords ({\ttfamily PIRNT} instead of {\ttfamily PRINT})
	\item Missing or extra quotation marks
	\item Missing or mismatched parentheses
	\item Incorrect punctuation
	\item Using a reserved word as a variable name
\end{itemize}
Check your line carefully for typos.

\subsection*{?TYPE MISMATCH ERROR}
\index{Error Messages!TYPE MISMATCH}

You tried to use a number where a string was expected, or a string where a
number was expected. In BASIC, string variables end with a {\ttfamily \$} sign
(like {\ttfamily A\$}), and numeric variables don't (like {\ttfamily A}).
Make sure you are using the right type of variable.

\subsection*{?UNDEF'D FUNCTION ERROR}
\index{Error Messages!UNDEF'D FUNCTION}

You tried to use a function defined with {\ttfamily DEF FN} before the
{\ttfamily DEF FN} statement was executed. Make sure your function
definitions appear before the {\ttfamily FN} calls in your program.

\subsection*{?UNDEF'D STATEMENT ERROR}
\index{Error Messages!UNDEF'D STATEMENT}

A {\ttfamily GOTO}, {\ttfamily GOSUB}, or {\ttfamily RUN} statement refers
to a line number that doesn't exist in your program. Check the line number
and make sure the target line exists. Use {\ttfamily LIST} to see all the
line numbers in your program.

\subsection*{?VERIFY ERROR}
\index{Error Messages!VERIFY}

The program on the SD card does not match the program currently in memory.
This occurs when you use the {\ttfamily VERIFY} command to check if a program
was saved correctly. If you get this error, try saving the program again.

%----------------------------------------------------------------------------------------
\section*{Tips for Debugging}
%----------------------------------------------------------------------------------------

When you encounter an error:

\begin{enumerate}
	\item Note the error message and line number
	\item Use {\ttfamily LIST} followed by the line number to see the problem line:\\
	      {\ttfamily LIST 100} shows line 100
	\item Check for common mistakes: typos, missing quotes, wrong variable types
	\item Fix the line by retyping it with the correct syntax
	\item Use {\ttfamily RUN} to test your fix
\end{enumerate}

Remember, error messages are your friends---they tell you exactly where to
look for problems in your program!
